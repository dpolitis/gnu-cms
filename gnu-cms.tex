%!TEX options = -shell-escape

\documentclass[12pt]{report}
\usepackage[a4paper,twoside,top=20mm,bottom=20mm,inner=30mm,outer=25mm]{geometry}
\usepackage[utf8]{inputenc}
\usepackage[greek,english]{babel}
\usepackage[scaled=0.86]{couriers}
\usepackage[toc,page,title,titletoc]{appendix}
\usepackage[pdfpagelabels,unicode]{hyperref}
\usepackage{bookmark}
\usepackage[fixlanguage]{babelbib}
\selectbiblanguage{greek}
\usepackage{titlesec}
\usepackage{etoolbox}
\usepackage{graphicx}
\usepackage{array}
\usepackage{amsmath}
\usepackage{minted}
\usepackage{subcaption}
\captionsetup{compatibility=false}
\graphicspath{ {images/} }
\usepackage[noend]{algpseudocode}
\usepackage{algorithm}
\usepackage{afterpage}
\usepackage{comment}

\newcommand\blankpage{%
    \null
    \thispagestyle{empty}%
    \addtocounter{page}{-1}%
    \newpage}

\hypersetup{
  colorlinks=true,
  % linkcolor=green,
  citecolor=red,
  % filecolor=blue,
  urlcolor=blue,
  % pdftitle=,
  % pdfauthor=,
  % pdfsubject=,
  % pdfkeywords=
}

\setcounter{secnumdepth}{3}
\setcounter{tocdepth}{3}

\titleformat{\chapter}
  {\normalfont\LARGE\bfseries}{\thechapter}{1em}{}
\titlespacing*{\chapter}{0pt}{3.5ex plus 1ex minus .2ex}{2.3ex plus .2ex}

\makeatletter
\patchcmd\@resets@pp{%
  \def\Hy@chapapp{\appendixname }%
}{%
  \def\Hy@chapapp{appendix}%
}{}{\errmessage{Cannot patch \string\@resets@pp}}
\patchcmd\@resets@ppsub{%
  \def\Hy@chapapp{\appendixname }%
}{%
  \def\Hy@chapapp{appendix}%
}{}{\errmessage{Cannot patch \string\@resets@pp}}
\makeatother

\addto{\captionsgreek}{\renewcommand{\appendixpagename}{Παραρτήματα}}
\addto{\captionsgreek}{\renewcommand{\appendixtocname}{Παραρτήματα}}
\addto{\captionsgreek}{\renewcommand{\appendixname}{Παράρτημα}}

\begin{document}
\selectlanguage{greek}

\hypersetup{pageanchor=false}

\begin{titlepage}
  \centering
  \includegraphics[width=0.15\textwidth]{pyrforos}\par\vspace{1cm}
  {\scshape\LARGE Εθνικό Μετσόβιο Πολυτεχνείο\\
  Σχολή Ηλεκτρολόγων Μηχανικών και Μηχανικών Η/Υ\par}
  \vspace{1cm}
  {\scshape\Large Εργασία στο Μεταπτυχιακό Μάθημα\\
  Τεχνολογίες Λογισμικού Για Παροχή Υπηρεσιών σε Επικοινωνιακά Δίκτυα\par}
  \vspace{1.5cm}
  {\Large\bfseries Συστήματα Διαχείρισης Περιεχομένου Ανοιχτού Κώδικα\par}
  \vspace{2cm}
  {\large Δημήτριος Πολίτης (ΥΔ)\par}
  \vfill
  Επιβλέπων \par
  Καθ. Ευστάθιος Συκάς

  \vfill

% Bottom of the page
  {\large \today\par}
  \afterpage{\blankpage}
\end{titlepage}

\tableofcontents
\thispagestyle{empty}

\listoftables
\thispagestyle{empty}

\listoffigures
\thispagestyle{empty}

\begin{abstract}
Στο παρόν μελετώνται 

\vspace{10mm}

\noindent \textbf{Λέξεις κλειδιά:} Συστήματα Διαχείρισης Περιεχομένου, Ανοιχτός Κώδικας, Εξυπηρετητής Ιστοσελίδων Διαδίκτυο.
\end{abstract}

\hypersetup{pageanchor=true}
\clearpage
\pagenumbering{arabic}

\chapter{Εισαγωγή}\label{ch1}
\section{Εισαγωγή}

\section{Συμπεράσματα}
Στο παρόν, παρουσιάστηκε ένα μοντέλο προσομοίωσης του πρωτοκόλλου πολλαπλής πρόσβασης μέσου (\textlatin{MAC}) τύπου \textlatin{slotted aloha}. Τα αποτελέσματα της προσομοίωσης έδειξαν τιμές πολύ κοντά στις θεωρητικά αναμενόμενες. Συγκεκριμένα, η αύξηση των μνημών προσωρινής αποθήκευσης πακέτων (\textlatin{receiving buffers}) στους σταθμούς, είχε θετική επίδραση στην απόδοση του συστήματος \textlatin{S}, η οποία όμως εμφανίζονταν φθίνουσα, καθώς αυξάνονταν το πλήθος των σταθμών \textlatin{M}. Επίσης, η αύξηση του πλήθους των \textlatin{receiving buffers} πάνω από δύο, δεν είχε κάποια σημαντική επίδραση στην αποδοτικότητα του συστήματος. Αντιθέτως, η αύξηση των διαύλων επικοινωνίας μεταξύ των κόμβων είχε σημαντικά θετική επίδραση στην διεκπεραιωτική δυνατότητα του συστήματος και τη μείωση της καθυστέρησης, ανεξάρτητα από το πλήθος των σταθμών.

Τα παραπάνω καταδεικνύουν ότι η βέλτιστη απόδοση του πρωτοκόλλου επιτυγχάνεται όταν οι σταθμοί έχουν δύο το πλήθος \textlatin{receiving buffers} και ικανό αριθμό καναλιών - διαύλων επικοινωνίας μεταξύ τους (ο οποίος μπορεί να περιορίζεται από τεχνικούς περιορισμούς ή περιορισμούς κόστους). Στην περίπτωση των ασύρματων δικτύων αυτό μπορεί να μεταφραστεί: είτε σε ικανό αριθμό πομποδεκτών ανά σταθμό, οι οποίοι θα λειτουργούν σε διαφορετικές συχνότητες για λόγους αποφυγής παρεμβολών, είτε σε κατάλληλο σχήμα πολύπλεξης (\textlatin{TDM, FDM}), επαναχρησιμοποίηση φάσματος με αναπήδηση συχνότητας κ.α. Κατά αντιστοιχία, στα οπτικά δίκτυα είναι δυνατή η χρησιμοποίηση πολύτροπων οπτικών ινών και εκπομπή των δεδομένων σε διαφορετικά μήκη κύματος για την ταυτόχρονη χρησιμοποίηση του μέσου.

\begin{appendices}

\end{appendices}

\appendix

\bibliographystyle{babplain}
\bibliography{gnu-cms}

\end{document}