%!TEX options = -shell-escape

\documentclass[12pt]{report}
\usepackage[a4paper,twoside,top=20mm,bottom=20mm,inner=30mm,outer=25mm]{geometry}
\usepackage[utf8]{inputenc}
\usepackage[greek,english]{babel}
\usepackage[scaled=0.86]{couriers}
\usepackage[toc,page,title,titletoc]{appendix}
\usepackage[pdfpagelabels,unicode]{hyperref}
\usepackage{bookmark}
\usepackage[fixlanguage]{babelbib}
\selectbiblanguage{greek}
\usepackage{titlesec}
\usepackage{etoolbox}
\usepackage{graphicx}
\usepackage{array}
\usepackage{amsmath}
\usepackage{minted}
\usepackage{subcaption}
\captionsetup{compatibility=false}
\graphicspath{ {images/} }
\usepackage[noend]{algpseudocode}
\usepackage{algorithm}
\usepackage{afterpage}
\usepackage{comment}
\usepackage{longtable}

\newcommand\blankpage{%
    \null
    \thispagestyle{empty}%
    \addtocounter{page}{-1}%
    \newpage}

\hypersetup{
  colorlinks=true,
  % linkcolor=green,
  citecolor=red,
  % filecolor=blue,
  urlcolor=blue,
  % pdftitle=,
  % pdfauthor=,
  % pdfsubject=,
  % pdfkeywords=
}

\setcounter{secnumdepth}{3}
\setcounter{tocdepth}{3}

\titleformat{\chapter}
  {\normalfont\LARGE\bfseries}{\thechapter}{1em}{}
\titlespacing*{\chapter}{0pt}{3.5ex plus 1ex minus .2ex}{2.3ex plus .2ex}

\makeatletter
\patchcmd\@resets@pp{%
  \def\Hy@chapapp{\appendixname }%
}{%
  \def\Hy@chapapp{appendix}%
}{}{\errmessage{Cannot patch \string\@resets@pp}}
\patchcmd\@resets@ppsub{%
  \def\Hy@chapapp{\appendixname }%
}{%
  \def\Hy@chapapp{appendix}%
}{}{\errmessage{Cannot patch \string\@resets@pp}}
\makeatother

\addto{\captionsgreek}{\renewcommand{\appendixpagename}{Παραρτήματα}}
\addto{\captionsgreek}{\renewcommand{\appendixtocname}{Παραρτήματα}}
\addto{\captionsgreek}{\renewcommand{\appendixname}{Παράρτημα}}

\begin{document}
\selectlanguage{greek}

\hypersetup{pageanchor=false}

\begin{titlepage}
  \centering
  \includegraphics[width=0.15\textwidth]{pyrforos}\par\vspace{1cm}
  {\scshape\LARGE Εθνικό Μετσόβιο Πολυτεχνείο\\
  Σχολή Ηλεκτρολόγων Μηχανικών και Μηχανικών Η/Υ\par}
  \vspace{1cm}
  {\scshape\Large Εργασία στο Μεταπτυχιακό Μάθημα\\
  Τεχνολογίες Λογισμικού Για Παροχή Υπηρεσιών σε Επικοινωνιακά Δίκτυα\par}
  \vspace{1.5cm}
  {\Large\bfseries Συστήματα Διαχείρισης Περιεχομένου Ανοιχτού Κώδικα\par}
  \vspace{2cm}
  {\large Δημήτριος Πολίτης (ΥΔ)\par}
  \vfill
  Επιβλέπων \par
  Καθ. Ευστάθιος Συκάς

  \vfill

% Bottom of the page
  {\large \today\par}
  \afterpage{\blankpage}
\end{titlepage}

\tableofcontents
\thispagestyle{empty}

\listoftables
\thispagestyle{empty}

\listoffigures
\thispagestyle{empty}

\begin{abstract}
Στο παρόν μελετώνται τα λογισμικά ανοιχτού κώδικα, τα οποία αφορούν σε διαχείριση περιεχομένου, με έμφαση στο \textlatin{web content} και τις δυναμικές ιστοσελίδες. Παρουσιάζονται αρχικά τα διαθέσιμα λογισμικά, τα πλεονεκτήματα και μειονεκτήματά τους και στη συνέχεια περιγράφεται αναλυτικά η διαδικασία δημιουργίας ενός ιστοτόπου \textlatin{Drupal} με τη χρήση αυτοματοποιημένων εργαλείων (\textlatin{vagrant}). Η εργασία και ο σχετικός κώδικας είναι διαθέσιμα από το \textlatin{\url{https://github.com/dpolitis/gnu-cms}}.

\vspace{10mm}

\noindent \textbf{Λέξεις κλειδιά:} Συστήματα Διαχείρισης Περιεχομένου, Ανοιχτός Κώδικας, Εξυπηρετητής Ιστοσελίδων, Διαδίκτυο.
\end{abstract}

\hypersetup{pageanchor=true}
\clearpage
\pagenumbering{arabic}

\chapter{Εισαγωγή}\label{ch1}
\section{Εισαγωγή}
Στην εποχή του διαδικτύου λένε ότι υπάρχει κάποιος ή κάτι όταν έχει ηλεκτρονική παρουσία σε αυτό. Μπορεί αυτό να ακούγεται εν μέρει υπερβολικό, αλλά η εικόνα που παρουσιάζει μια εταιρία ή ένα φυσικό πρόσωπο στο διαδίκτυο επηρεάζει σε μεγάλο βαθμό την φήμη και την αξιοπιστία του.

Η δημιουργία, η συντήρηση και η ανανέωση του δυναμικού περιεχομένου των ιστοτόπων αποτελεί ένα δύσκολο αντικείμενο, γι´αυτό και για την διευκόλυνση του προσωπικού, το οποίο ασχολείται συστηματικά με τις παραπάνω εργασίες, έχουν αναπτυχθεί ειδικά εργαλεία, τα οποία ονομάζονται Συστήματα Διαχείρισης Περιεχομένου.

\section{Συστήματα Διαχείρισης Περιεχομένου}
Τα Συστήματα Διαχείρισης Περιεχομένου (\textlatin{Content Management Systems, CMS}) είναι εργαλεία δημιουργίας και διαχείρησης ιστοτόπων. Οι εφαρμογές αυτές εξαλείφουν την ανάγκη συγγραφής κώδικα προγραμματισμού~\cite{linode}. Από τη στιγμή την οποία έχουν αναπτυχθεί εντός των υποδομών, επιτρέπουν σε προσωπικό με μη εξειδικευμένες γνώσεις να διαχειρίζεται το περιεχόμενο των ιστοσελίδων τους.
Η επίπονη διαδικασία ανάπτυξης κώδικα για την ανανέωση των ιστοσελίδων, αντικαθίσταται από μια φιλική προς το χρήστη, διεπαφή.

\subsection{Πλεονεκτήματα}
Υπάρχουν αρκετές δεκάδες διαθέσιμα λογισμικά αυτού του είδους και πολλά από αυτά είναι δωρεάν - ανοιχτού κώδικα. Κάθε \textlatin{CMS} έχει διαφορετικά χαρακτηριστικά, δυνατότητες ή διεπαφή χρήστη. Οι εφαρμογές αυτές είναι διαθέσιμες σχεδόν από τα τέλη της δεκαετίας του 1990 και συνεχίζουν να χρησιμοποιούνται από όλο και περισσότερους χρήστες.

Πολλές φορές, ακόμα και έμπειροι προγραμματιστές προτιμούν την ευκολία ενός \textlatin{CMS} από το να γράψουν τον κώδικα για έναν ιστότοπο από την αρχή. Η χρήση των \textlatin{CMS} διευκολύνει την ταχύτερη ανάπτυξη του ιστοτόπου, ακόμα και εντός λίγων ημερών. Παρέχει επίσης τη δυνατότητα παρακολούθησης των αλλαγών στον κώδικα με τη χρήση ενσωματωμένων εργαλείων \textlatin{version control}~\cite{wikipedia_2017:02}.

Όλο και περισσότερα \textlatin{CMS} είναι πλέον λογισμικά ανοιχτού κώδικα. Η ιδιότητά αυτή επιτρέπει στους χρήστες να αναπτύσσουν πρόσθετα (\textlatin{add-ons}) και να τεκμηριώνουν πληρέστερα τις λειτουργίες τους. Επίσης, είναι δυνατή η παροχή βοήθειας, ανταλλαγή απόψεων, σχολίων κτλ, μεταξύ χρηστών.

Πολλά από τα \textlatin{CMS} δημιουργούν την δομή της ιστοσελίδας, χρησιμοποιώντας θέματα. Τα θέματα βοηθούν στην καλή εμφάνιση της ιστοσελίδας, γεγονός το οποίο με τη σειρά του κάνει τους χρήστες να επιστρέφουν για νέο περιεχόμενο. Η αλλαγή θεμάτων και δομής της ιστοσελίδας είναι αρκετά εύκολη διαδικασία με τη χρήση \textlatin{CMS}. Επίσης πολλά από τα θέματα βασίζονται σε αρχεία \textlatin{CSS} ή \textlatin{HTML}, πράγμα το οποίο δίνει τη δυνατότητα στους προγραμματιστές να τα παραμετροποιούν σύμφωνα με τις ανάγκες τους και να διατηρούν μια σταθερή εμφάνιση σε όλες τις ιστοσελίδες. Τέλος, πολλά λογισμικά \textlatin{CMS} παρέχουν είτε δωρεάν επιπλέον θέματα, είτε επί πληρωμή, αυξάνοντας κατακόρυφα τις δυνατότητες των επιλογών.

Τα περισσότερα \textlatin{CMS} είναι φτιαγμένα ώστε να παρέχουν την απαραίτητη συμβατότητα με άλλα \textlatin{frameworks} ή \textlatin{standards}, αυξάνοντας την παραγωγικότητα και τις δυνατότητές τους~\cite{wikipedia_2017:02}.

Επίσης, τα \textlatin{CMS} έχουν τη δυνατότητα για αναβάθμιση όλων των λειτουργικών μερών τους με τη χρήση απλών οδηγών, φιλικών προς το χρήστη. Τις περισσότερες φορές παρέχεται και η δυνατότητα \textlatin{rollback}, εφόσον η διαδικασία δεν επιτευχθεί με επιτυχία.

Τα \textlatin{CMS} βοηθούν στην απλοποίηση των διαδικασιών και από την πλευρά των μη τεχνικών χρηστών, καθώς δεν απαιτούν κάποια εξειδικευμένη τεχνική γνώση για την ανάρτηση νέου περιεχομένου σε υπάρχουσες ιστοσελίδες, για παράδειγμα ιστολόγια.

Τέλος, τα \textlatin{CMS} παρέχουν ένα συγκεκριμένο επίπεδο ασφαλείας έναντι σε επιθέσεις εκμετάλευσης αδυναμιών στον κώδικα των ιστοσελίδων, καθώς πρόκειται για λογισμικά, τα οποία περνούν από εκτεταμένους ελέγχους κατά την ανάπτυξή τους, πριν την κυκλοφορία τους. Αυτό ισχύει τόσο για τα λογισμικά ανοιχτού κώδικα - τα οποία υποστηρίζονται συνήθως από μια μεγάλη κοινότητα χρηστών - όσο και τα εμπορικά \textlatin{CMS}. Στα παραπάνω αξίζει να προστεθεί το γεγονός ότι, τα περισσότερα \textlatin{CMS} παρέχουν ενσωματωμένους μηχανισμούς παρακολούθησης και εξαγωγής συμπερασμάτων από τα αρχεία καταγραφής.

\subsection{Μειονεκτήματα}
Τα λογισμικά \textlatin{CMS} έχουν ένα κύριο μειονέκτημα το οποίο κυρίως αφορά στον περιορισμό των δυνατοτήτων παραμετροποίησης μιας εφαρμογής, όταν αυτή αναμένεται να λειτουργήσει υπό διαφορετικά περιβάλλοντα. Για παράδειγμα, όταν μια εφαρμογή αναγκαστεί να λειτουργήσει πίσω από ένα πλήθος \textlatin{firewall} η \textlatin{reverse proxy}. Το παραπάνω πρόβλημα συνήθως αντιμετωπίζεται με \textlatin{plugins} στα πιο δημοφιλή \textlatin{CMS} (για παράδειγμα \textlatin{wordpress}).

Ένα ακόμα μειονέκτημα, το οποίο αφορά συνήθως στους διαχειριστές συστημάτων, είναι ότι τα περισσότερα \textlatin{CMS} απαιτούν ειδικές ρυθμίσεις - πολλές φορές μη-ασφαλείς ή εγκατάσταση ειδικών εργαλείων στους παραγωγικούς \textlatin{server} (για παράδειγμα \textlatin{drush - drushx} για το \textlatin{drupal}).

\section{Κατηγοριοποίηση των \textlatin{CMS}}
Τα \textlatin{CMS} μπορούν να κατηγοριοποιηθούν με βάση το περιεχόμενο, το οποίο πραγματεύονται, σε κατηγορίες όπως παρακάτω:

\paragraph{\textlatin{CMS} γενικής χρήσης} Τα \textlatin{CMS} αυτού του τύπου χρησιμοποιούνται για την κατασκευή ιστοσελίδων γενικού περιεχομένου. Συνήθως περιλαμβάνουν ενημερωτικό ή προωθητικό περιεχόμενο, χωρίς να περιλαμβάνουν αγοραπωλησίες. Τα τρία πιο διαδεδομένα σύγχρονα \textlatin{CMS} είναι λογισμικά ανοιχτού κώδικα, το καθένα με τα πλεονεκτήματα και τις αδυναμίες του (\textlatin{Drupal, Joomla, Wordpress}).

\paragraph{\textlatin{CMS} ηλεκτρονικού εμπορίου} Συνήθως χρησιμοποιούνται για τη διαχείριση ηλεκτρονικών καταστημάτων. Αποτελούν στις περισσότερες των περιπτώσεων ολοκληρωμένα συστήματα διαχείρησης ηλεκτρονικών αγορών με λειτουργικότητες που εκτείνονται από την καταχώρηση της επιθυμίας ενός πελάτη στο καλάθι αγορών, μέχρι την εκδοσή τιμολογίων, αποστολή στον πελάτη, ακόμα και παρακολούθηση αποθεμάτων.

\paragraph{\textlatin{CMS} δημοπρασιών} Αποτελούν συναφή κατηγορία με την παραπάνω με τη διαφορά ότι αφορούν ηλεκτρονικές δημοπρασίες. Πολλές φορές μπορούν κάποια \textlatin{CMS} να ανήκουν και στις δυο κατηγορίες (π.χ. \textlatin{PHP-ProBid}).

\paragraph{\textlatin{CMS} διαχείρησης πολυμέσων (\textlatin{DAMS})} Αφορούν σε \textlatin{CMS} τα οποία χειρίζονται \textlatin{vlogs} ή σελίδες με πολυμέσα γενικότερα.

Στο παρόν παρουσιάζονται τα κυριότερα \textlatin{CMS} γενικού περιεχομένου, ανοιχτού κώδικα με έμφαση στα δημοφιλέστερα, με βάση το μερίδιο αγοράς του καθενός.

\chapter{Παρουσίαση των \textlatin{Open Source CMS}}\label{ch2}
Παρακάτω παρουσιάζονται τα κυριότερα \textlatin{Open Source CMS} με μια συνοπτική περιγραφή και τα πλεονεκτήματα - μειονεκτήματά του. Επίσης για κάθε \textlatin{CMS} παρατίθεται και μια μικρή αποτύπωση του περιβάλλοντος αναρτήσης περιεχομένου.

\section{\textlatin{ExpressionEngine}}
Το \textlatin{ExpressionEngine} είναι ένα εμπορικό προϊόν το οποίο έχει στηθεί πάνω σε ένα \textlatin{opensource project}, Δίνοντας έτσι τα καλύτερα χαρακτηριστικά και από τους δυο τρόπους ανάπτυξης λογισμικού. Ως ένα εμπορικό λογισμικό το \textlatin{ExpressionEngine} υποστηρίζεται από μια ομάδα αφοσιωμένων \textlatin{developers} και εξειδικευμένων τεχνικών.
\begin{figure}[H]
\centering
\includegraphics[width=0.9\textwidth, height=5cm]{ExpressionEngine-gray}
\caption{\textlatin{ExpressionEngine CMS}}
\label{fig:expr_eng}
\end{figure}

\section{\textlatin{Textpattern}}
Το \textlatin{Textpattern} είναι ένα κομψό \textlatin{CMS} το οποίο διατίθεται δωρεάν και είναι ανοιχτού κώδικα. Οι σχεδιαστές, οι \textlatin{developers} αλλά και οι \textlatin{blogers} μπορούν να βοηθηθούν από την ευελιξία και την επεκτασιμότητα του. Αποτελείται από μια εξελιγμένη μηχανή που μπορεί να χρησιμοποιηθεί για την υποστήριξη οποιουδήποτε τύπου ιστοσελίδας απαιτηθεί.
\begin{figure}[H]
\centering
\includegraphics[width=0.9\textwidth, height=5cm]{Textpattern-gray}
\caption{\textlatin{Textpattern CMS}}
\label{fig:text_patrn}
\end{figure}

\section{\textlatin{Symphony CMS}}
Το \textlatin{Symphony} είναι ένα \textlatin{content management system (CMS)} το οποίο δίνει τη δυνατότητα στους χρήστες να δημιουργούν και να διαχειρίζονται ιστοσελίδες και \textlatin{web applications} ανεξαρτήτως μεγέθους – από τα απλούστερα \textlatin{blogs} εώς και τις πιο περίπλοκες ιστοσελίδες ειδήσεων ή κοινωνικών δικτύων.
\begin{figure}[H]
\centering
\includegraphics[width=0.9\textwidth, height=5cm]{symphony_cms-gray}
\caption{\textlatin{Symphony CMS}}
\label{fig:symphony_cms}
\end{figure}

\section{\textlatin{CMS Made Simple}}
Το \textlatin{CMS Made Simple™} είναι ένα λογισμικό ανοικτού κώδικα, το οποίο διανεμήθηκε στην πρώτη του έκδοση τον Ιούλιο του 2004. Στηρίζεται στην γλώσσα \textlatin{PHP} και παρέχει στους \textlatin{developers} έναν απλό τρόπο για τη δημιουργία και τη διαχείριση μικρών σε έκταση ιστοσελίδων με στατικό ή δυναμικό κώδικα~\cite{gongea}.
\begin{figure}[H]
\centering
\includegraphics[width=0.9\textwidth, height=5cm]{cms_made_simple-gray}
\caption{\textlatin{CMS Made Simple CMS}}
\label{fig:made_simpl}
\end{figure}

\section{\textlatin{Concrete5}}
Το \textlatin{Concrete5} δίνει τη δυνατότητα στους \textlatin{developers} να δημιουργήσουν μια ιστοσελίδα κυριολεκτικά σε δευτερόλεπτα. Παρέχει ένα εύκολο περιβάλλον διαχείρησης περιεχομένου, καθοριζόμενου μέσω \textlatin{point and click}, το οποίο επιτρέπει και σε μη τεχνικούς χρήστες να το χειρίζονται και να μεταβάλλουν το περιεχόμενο.
\begin{figure}[H]
\centering
\includegraphics[width=0.9\textwidth, height=5cm]{Concrete5-gray}
\caption{\textlatin{Concrete5 CMS}}
\label{fig:concrete5}
\end{figure}

\section{\textlatin{Website Baker}}
Το \textlatin{Website Baker} επιτρέπει τη δημιουργία \textlatin{templates} ιστοσελίδων μέσα σε λίγα λεπτά. Το \textlatin{CMS} στηρίζεται πάνω σε \textlatin{(X)HTML, CSS} και \textlatin{jQuery}. Η επαύξηση των λειτουργιών του γίνεται με τη χρήση \textlatin{droplets}, τα οποία είναι κομμάτια κώδικα \textlatin{PHP} και τα οποία μπορούν να εισαχθούν σχεδόν οπουδήποτε στο \textlatin{CMS}.
\begin{figure}[H]
\centering
\includegraphics[width=0.9\textwidth, height=5cm]{websitebaker-gray}
\caption{\textlatin{Website Baker CMS}}
\label{fig:site_baker}
\end{figure}

\section{\textlatin{Umbraco}}
Το \textlatin{Umbraco} είναι ένα δωρεάν λογισμικό ανοιχτού κώδικα, το οποίο είναι βασισμένο πάνω στο \textlatin{Microsoft .NET Framework}. Είναι αρκετά εύχρηστο, απλό, κατανοητό και πλήρως επεκτάσιμο, με τη χρήση \textlatin{industry-standard} γλωσσών, όπως \textlatin{HTML, CSS, jQuery} και \textlatin{C\#}. Το \textlatin{Umbraco} είναι εξίσου ευέλικτο και δυνατό είτε χρησιμοποιείται από πεπειραμένους \textlatin{developers}, είτε από χρήστες που μόλις ξεκινούν να το χρησιμοποιούν.
\begin{figure}[H]
\centering
\includegraphics[width=0.9\textwidth, height=5cm]{umbraco-gray}
\caption{\textlatin{Umbraco CMS}}
\label{fig:umbraco}
\end{figure}

\section{\textlatin{Contao}}
Το \textlatin{Contao} έχει ένα αρκετά εύχρηστο και πλήρως πλοηγήσιμο περιβάλλον, το οποίο χρησιμοποιεί τεχνολογίες \textlatin{Ajax / Web 2.0} για βελτιστοποιημένη χρηστικότητα. Επίσης περιλαμβάνει πολλαπλά θέματα αλλά και διασυνδέσεις με πληθώρα γλωσσών \textlatin{backend}. Παρέχει, πέραν των άλλων, ισχυρό σύστημα διαχείρησης δικαιωμάτων, \textlatin{versioning} κώδικα και διαχείριση αλλαγών (\textlatin{undo management}), εξελιγμένες επιλογές αναζήτησης και ταξινόμησης, καθώς και τη δυνατότητα εγκατάστασης ενημερώσεων εν θερμώ, τα οποία το καθιστούν ένα από τα πιο ολοκληρωμένα \textlatin{content management system (CMS)}. Το \textlatin{front end} του είναι 100\% βασισμένο στη χρήση \textlatin{templates}, και ο παραγόμενος κώδικας είναι πλήρως αναγνώσιμος και εναρμονισμένος με τις προδιαγραφές του \textlatin{W3C/WAI}.
\begin{figure}[H]
\centering
\includegraphics[width=0.9\textwidth, height=5cm]{Contao-gray}
\caption{\textlatin{Contao CMS}}
\label{fig:contao}
\end{figure}

\section{\textlatin{Plone CMS}}
Το \textlatin{Plone} είναι ένα \textlatin{CMS}, το οποίο είναι προσανατολισμένο κυρίως στη χρήση \textlatin{applications}, ενώ ένα συνηθισμένο \textlatin{CMS} είναι συνήθως προσανατολισμένο στη δημιουργία σελίδων. Εάν οι χρήστες ενός συστήματος θέλουν να προσθέσουν, διορθώσουν ή αφαιρέσουν περιεχόμενο, το οποίο προϋποθέτει διεργασίες ή περίπλοκους τύπους δεδομένων βασισμένους στη φυσική διάρθρωση ενός οργανισμού, τότε το \textlatin{Plone} είναι ίσως η καταλληλότερη επιλογή.
\begin{figure}[H]
\centering
\includegraphics[width=0.9\textwidth, height=5cm]{plone_cms-gray}
\caption{\textlatin{Plone CMS}}
\label{fig:plone}
\end{figure}

\section{\textlatin{XOOPS}}
Το \textlatin{XOOPS} είναι ένα \textlatin{web application platform} το οποίο στηρίζεται στη γλώσσα \textlatin{PHP} και την ύπαρξη μιας βάσης δεδομένων \textlatin{MySQL}. Η αντικειμενοστραφής του σχεδίαση το κάνει ιδανικό εργαλείο για την ανάπτυξη ιστοσελίδων ποικίλου μεγέθους, \textlatin{corporate portals, weblogs} κ.α.
\begin{figure}[H]
\centering
\includegraphics[width=0.9\textwidth, height=5cm]{xoops-gray}
\caption{\textlatin{XOOPS CMS}}
\label{fig:xoops}
\end{figure}

\section{\textlatin{MODX}}
Το \textlatin{MODx} παρέχει ένα ασφαλές περιβάλλον διαχείρησης, με το οποίο είναι δυνατό να δημιουργήσει κανείς μια ιστοσελίδα με ασφαλή τρόπο. Για παράδειγμα παρέχεται ένα σύστημα διαχωρισμού των χρηστών και διαχειριστών της ιστοσελίδας. Όσον αφορά τη διαχείριση περιεχομένου, παρέχεται η δυνατότητα κλωνοποίησης εγγράφων, δεδομένων ακόμα και ολόκληρων φακέλων σε απεριόριστο βάθος.
\begin{figure}[H]
\centering
\includegraphics[width=0.9\textwidth, height=5cm]{modx-gray}
\caption{\textlatin{MODX CMS}}
\label{fig:modx}
\end{figure}

\section{\textlatin{Silverstripe}}
Το \textlatin{SilverStripe} είναι ένα αρκετά απλό \textlatin{CMS} ανοιχτού λογισμικού, το οποίο χρησιμοποιείται από πολλούς επαγγελματίες \textlatin{developers} για τη δημιουργία δυναμικού περιεχομένου. Το γεγονός ότι είναι πολύ εύχρηστο, το καθιστά ιδανική επιλογή και για μη τεχνικούς-χρήστες, οι οποίοι επιθυμούν να δημιουργήσουν εύκολα και γρήγορα έναν ιστότοπο.
\begin{figure}[H]
\centering
\includegraphics[width=0.9\textwidth, height=5cm]{silverstripe-gray}
\caption{\textlatin{Silverstripe CMS}}
\label{fig:silverstripe}
\end{figure}

\section{\textlatin{PyroCMS}}
Το \textlatin{PyroCMS} παρέχει καλή εμφάνιση και είναι εύκολο στη χρήση του. Εκτός αυτού, χρησιμοποιεί ένα σύστημα έξυπνου \textlatin{caching} για να αυξάνει την ταχύτητα απόκρισης. Είναι εύκολα επεκτάσιμο με αρθρώματα και πρόσθετα, ενώ το γεγονός ότι στηρίζεται στο \textlatin{CodeIgniter framework}, το καθιστά εύκολα τροποποιήσιμο όσον αφορά στην εμφάνιση του ιστοτόπου με τη χρήση θεμάτων, τα οποία είναι απλός κώδικας \textlatin{HTML}.
\begin{figure}[H]
\centering
\includegraphics[width=0.9\textwidth, height=5cm]{pyrocms-gray}
\caption{\textlatin{PyroCMS CMS}}
\label{fig:pyrocms}
\end{figure}

\section{\textlatin{GetSimple CMS}}
Το \textlatin{GetSimple} είναι ένα ελαφρύ \textlatin{CMS} το οποίο στηρίζεται στη χρήση τεχνολογιών \textlatin{XML}. Παρόλο που είναι ελαφρύ, έχει όλες τις δυνατότητες που θα χρειαζόταν κάποιος για την δημιουργία και συντήρηση ενός ιστοτόπου μιας μικρής ή μεσαίας επιχείρησης.
\begin{figure}[H]
\centering
\includegraphics[width=0.9\textwidth, height=5cm]{getsimplecms-gray}
\caption{\textlatin{GetSimple CMS}}
\label{fig:get_simpl}
\end{figure}

\section{\textlatin{FuelCMS}}
Το \textlatin{FUEL CMS} είναι άλλο ένα \textlatin{CMS} το οποίο στηρίζεται στο \textlatin{CodeIgniter framework} και το οποίο είναι ελαφρύ, εξαιρετικά παραμετροποιήσιμο και επεκτάσιμο. Η χρήση του \textlatin{CodeIgniter framework} απαιτεί τη συγγραφή κώδικα και έτσι το \textlatin{FUEL CMS} απευθύνεται κυρίως σε επαγγελματίες \textlatin{developers}, χωρίς αυτό να σημαίνει ότι ένας απλός χρήστης δε μπορεί να το χρησιμοποιήσει, απλά για να ανεβάσει νέο περιεχόμενο.
\begin{figure}[H]
\centering
\includegraphics[width=0.9\textwidth, height=5cm]{fuelcms-gray}
\caption{\textlatin{FuelCMS CMS}}
\label{fig:fuelcms}
\end{figure}

\section{\textlatin{Drupal}}
To \textlatin{Drupal} είναι ένα ισχυρό και εξελιγμένο \textlatin{CMS framework} το οποίο στηρίζει την λειτουργία του στην γλώσσα \textlatin{php} και στην ύπαρξη μιας βάσης δεδομένων, όπως η \textlatin{MySQL}. To  \textlatin{Drupal} παρέχει ένα ευέλικτο περιβάλλον, το οποίο χρησιμοποιείται για τη διαχείριση ιστοτόπων διαφόρων τύπων και προφίλ. Μπορεί να χρησιμοποιηθεί για τη δημιουργία πλούσιων σε περιεχόμενο ιστοσελίδων με διαδραστικό περιεχόμενο, όπως φόρουμ, \textlatin{blogs} χρηστών και υπηρεσίες ανταλλαγής προσωπικών μηνυμάτων.

To \textlatin{Drupal} χρησιμοποιείται σε αρκετούς γνωστούς διαδικτυακούς τόπους, όπως για παράδειγμα το \textlatin{\url{http://www.weather.com}}. Καθώς βρίσκεται σε ενεργό κύκλο ανάπτυξης, αναβαθμίζεται αρκετά συχνά (σχεδόν κάθε 2 - 4 μήνες). Η εκτεταμένη κοινότητα χρηστών παρακολουθεί τα σχετικά συνέδρια, τα οποία λαμβάνουν χώρα κάθε 2 χρόνια σε Ευρώπη και Αμερική.

Η δύναμη του \textlatin{Drupal} βρίσκεται στην καλά οργανωμένη δομή του. Το \textlatin{Drupal} ξεκινά με ένα βασικό σύνολο αρχείων, το οποίο μπορεί να εμπλουτισθεί στη συνέχεια με διάφορα πρόσθετα, όπως θέματα ή αρθρώματα, που αυξάνουν τη λειτουργικότητά του. Ο αρχικός πυρήνας των αρχείων μπορεί να έχει ένα βασικό σετ από πρόσθετα ή θέματα, τα οποία μπορούν να τροποποιηθούν κατά βούληση.

Πέρα από τις παραπάνω παραμετροποιήσεις, υπάρχει διαθεσιμότητα σε σύνολα από πρόσθετα, τα οποία προορίζονται για συγκεκριμένη χρήση και συνήθως έρχονται με τη μορφή \textlatin{Drupal} διανομών. Διατίθενται για παράδειγμα διανομές για απλές εταιρικές ιστοσελίδες, για ενημερωτικού τύπου - πολυμεσικές ή ακόμα και διανομές γενικής χρήσης \textlatin{community-based}~\cite{opensource_cms_demos_and_information}.
\begin{figure}[H]
\centering
\includegraphics[width=0.9\textwidth, height=7cm]{drupal-gray}
\caption{\textlatin{Drupal CMS}}
\label{fig:drupal}
\end{figure}

\section{\textlatin{Joomla}}
Το \textlatin{Joomla} αναπτύσσεται ενεργά από το 2005 και χρησιμοποιείται σε πολύ γνωστές ιστοσελίδες όπως τις \textlatin{eBay, General Electric, Ikea} κ.α. Το \textlatin{Joomla} συγκριτικά με το \textlatin{Drupal}, το οποίο έχει περισσότερα πρόσθετα και θέματα, εμφανίζεται και αυτό αρκετά ενισχυμένο σε πλήθος πρόσθετων και επεκτάσεων. Εκτός αυτού, η βάση χρηστών και των δύο \textlatin{CMS} είναι αρκετά εκτεταμένη και έτσι είναι αρκετά εύκολη η εύρεση οδηγιών στο διαδίκτυο για αντιμετώπιση διαφόρων προβλημάτων.

Το \textlatin{Joomla} απευθύνεται κυρίως σε χρήστες μέσου επιπέδου γνώσεων~\cite{pixelmedia}. Διαθέτει εργαλεία για την συνεργατική ανάπτυξη ενός ιστοτόπου, αλλά και για κεντρικοποιημένη διαχείριση της διαδικασίας ανάπτυξης. Η διαδικασία εγκατάστασης δεν είναι ιδιαίτερα δύσκολη, αλλά η δημιουργία περιεχομένου δεν είναι τόσο εύκολη σε σχέση με το επόμενο προς εξέταση \textlatin{CMS}, το \textlatin{Wordpress}.
\begin{figure}[H]
\centering
\includegraphics[width=0.9\textwidth, height=7cm]{joomla-gray}
\caption{\textlatin{Joomla CMS}}
\label{fig:joomla}
\end{figure}

\section{\textlatin{Wordpress}}
Το \textlatin{Wordpress} δημιουργήθηκε αρχικά ως μια πλατφόρμα για τη δημιουργία και συντήρηση \textlatin{blogs}. Είναι μέχρι σήμερα, χωρίς αμφιβολία η ευκολότερη στη χρήση πλατφόρμα και η πιο δημοφιλής. Η ανάπτυξή της ξεκίνησε το 2003 και είναι ενεργή μέχρι σήμερα. Χρησιμοποιείται ήδη σε 60 εκατομμύρια περίπου ιστότοπους με ρυθμό αύξησης περίπου 100.000 νέους ιστότοπους ανά ημέρα.

Το \textlatin{Wordpress} είναι καλύτερο για την δημιουργία στατικού περιεχομένου. Αυτό δε σημαίνει όμως ότι δε μπορεί να χρησιμοποιηθεί και για δυναμικό ή περίπλοκο περιεχόμενο, καθώς η μεγάλη βάση χρηστών του έχει προσθέσει πλήθος λειτουργιών. Σήμερα η ανάπτυξή του καθοδηγείται ενεργά από την εταιρία \textlatin{Automattic}, με την απήχησή του να γίνεται όλο και μεγαλύτερη.
\begin{figure}[H]
\centering
\includegraphics[width=0.9\textwidth, height=7cm]{wordpress-gray}
\caption{\textlatin{Wordpress CMS}}
\label{fig:wordpress}
\end{figure}

\section{Σύγκριση των πιο Δημοφιλών \textlatin{CMS}}
Στα προηγούμενα αναφέρθηκαν συνοπτικά τα πιο γνωστά \textlatin{CMS} ανοιχτού κώδικα. Στο παρόν αναλύονται τα τρία πιο δημοφιλή \textlatin{CMS} με τα πλεονεκτήματα και μειονεκτήματά τους και δίνεται ένας συνοπτικός πίνακας με τα πιο σημαντικά χαρακτηριστικά τους.

\subsection{\textlatin{WordPress} το πιο Δημοφιλές \textlatin{CMS}}
Παρά την ταπεινή του καταγωγή από το χώρο του \textlatin{bloging}, το \textlatin{WordPress} έχει επί της ουσίας κατακτήσει την αγορά των \textlatin{CMS}, με τα ποσοστά χρήσης του να αγγίζουν το 40\% των ιστότοπων που βασίζονται σε κάποιο \textlatin{CMS}.

Τα πλεονεκτήματά του περιλαμβάνουν:
\begin{itemize}
\item Εύκολη εγκατάσταση. Πολλές εταιρίες παροχής υπηρεσιών \textlatin{hosting} παρέχουν αυτοματοποιημένα εργαλεία εγκατάστασης και παραμετροποίησης ενός νέου ιστότοπου σε \textlatin{Wordpress}. Αυτό σημαίνει ότι οποιοσδήποτε μπορεί να έχει ένα νέο ιστότοπο, έτοιμο προς παραμετροποίηση μέσα σε 5 λεπτά.

\item Δυνατότητες Παραμετροποίησης. Το \textlatin{WordPress} έχει σημαντικά μεγαλύτερο αριθμό πρόσθετων, θεμάτων και επιλογών παραμετροποίησης από οποιοδήποτε άλλο \textlatin{CMS}. Εξαιτίας της μεγάλης του δημοφιλίας, όλο και περισσότεροι προγραμματιστές αναπτύσσουν εργαλεία, δωρεάν ή με ένα μικρό κόστος, για την παραμετροποίηση του \textlatin{WordPress}. Τα εργαλεία αυτά δίνουν τη δυνατότητα για τη δημιουργία ενός ιστοτόπου με εμφάνιση που παραπέμπει σε αντίστοιχα επαγγελματικά με ένα μικρό κόστος γύρω στα 100 Ευρώ.

\item Δωρεάν. Εκτός από το ίδιο το \textlatin{WordPress}, το οποίο είναι δωρεάν, υπάρχει πληθώρα δωρεάν πρόσθετων και θεμάτων από τα οποία μπορεί κανείς να διαλέξει, ώστε να φτιάξει ένα νέο ιστότοπο. Αυτό ευνοεί τους νέους προγραμματιστές οι οποίοι θέλουν να δημιουργήσουν ένα ιστότοπο με το μικρότερο δυνατό κόστος. Επίσης παρέχεται η δυνατότητα δωρεάν φιλοξενίας του ιστοτόπου στο \textlatin{\url{https://www.wordpress.com}}.

\item Υποστήριξη από την κοινότητα. Οποιοδήποτε πρόβλημα και αν αντιμετωπίσει κάποιος, που θέλει να χρησιμοποιήσει το \textlatin{WordPress}, μπορεί να στραφεί σε μια πολυπληθή και ενεργή κοινότητα, σε αναζήτηση συμβουλών και λύσεων. Στα σχετικά φόρα μπορεί κανείς να μιλήσει με πεπειραμένους χρήστες και προγραμματιστές, οι οποίοι είναι πρόθυμοι να προσφέρουν τη βοήθειά τους δωρεάν και συνήθως σε μικρό χρονικό διάστημα.
\end{itemize}

Φυσικά το \textlatin{WordPress} παρουσιάζει και μειονεκτήματα.
\begin{itemize}
\item Το \textlatin{WordPress} θυσιάζει τη δυνατότητα ισχυρής παραμετροποίησης, ώστε να επιτύχει την ευκολία δημιουργίας ενός ιστότοπου για τους απλούς χρήστες. Κατά αυτό τον τρόπο δε μπορεί εύκολα κάποιος να κάνει ισχυρές αλλαγές στην εμφάνιση του ιστότοπου, εάν δεν είναι προγραμματιστής.

\item Το \textlatin{WordPress} εμφανίζεται πιο ευάλωτο σε επιθέσεις σε σχέση με τα υπόλοιπα \textlatin{CMS}, κυρίως λόγω της μεγαλύτερης βάσης χρηστών και του μεριδίου αγοράς που καταλαμβάνει.
\end{itemize}

\subsection{\textlatin{Drupal} πιο Σταθερό και Πολύπλοκο}
Το \textlatin{Drupal} είναι το τρίτο σε δημοφιλία \textlatin{CMS}, το οποίο χρησιμοποιείται για τη δημιουργία μικρών ή μεγάλων ιστότοπων.

Τα πλεονεκτήματά του περιλαμβάνουν:
\begin{itemize}
\item Τεχνικά εξελιγμένο. Το \textlatin{Drupal} είναι το πιο εξελιγμένο τεχνικά από τα τρία \textlatin{CMS}. Είναι πολύ καλό για τεχνικά καταρτισμένους χρήστες, οι οποίοι θέλουν να δουλέψουν πάνω στη δομή του ιστοτόπου.
\item Εξελιγμένη απόδοση. Το \textlatin{Drupal} έχει γενικότερα μικρότερους χρόνους απόκρισης και ταχύτερη φόρτωση και από τα υπόλοιπα δύο \textlatin{CMS}. Αυτό οφείλεται κυρίως στο γεγονός ότι δεν είναι τόσο απαιτητικό σε πόρους συστήματος. Φυσικά οι απαιτούμενοι πόροι αυξάνονται εφόσον αρχίσει κάποιος και φορτώνει με πρόσθετα το \textlatin{CMS}.
\item Ισχυρές δυνατότητες παραμετροποίησης. Το \textlatin{Drupal} είναι σχετικά εύκολα παραμετροποιήσιμο με πληθώρα πρόσθετων από τα οποία μπορεί να διαλέξει κανείς. Εκτός αυτού, οι προγραμματιστές έχουν τη δυνατότητα τροποποίησης των αρχείων κώδικα του \textlatin{Drupal}, εφόσον επιθυμούν να πραγματοποιήσουν μεγάλες αλλαγές στον ιστότοπο.
\item Δωρεάν. Είναι όπως και το \textlatin{WordPress} λογισμικό ανοιχτού κώδικα, με πληθώρα δωρεάν πρόσθετων και θεμάτων.
\end{itemize}

Το \textlatin{Drupal} είναι το πιο ισχυρό από άποψη δυνατοτήτων από τα υπόλοιπα δύο. Παρόλα αυτά παρουσιάζει τα παρακάτω μειονεκτήματα:
\begin{itemize}
\item Οι μεγάλες δυνατότητες παραμετροποίησης απαιτούν συνήθως περισσότερη προσπάθεια, από την πλευρά του προγραμματιστή αλλά και περισσότερο χρόνο εξοικείωσης.
\item Η χρήση του \textlatin{Drupal} απαιτεί μια σχετική γνώση \textlatin{HTML, PHP} και άλλων κοινών γλωσσών του διαδικτύου, κυρίως για λόγους επίλυσης τυχόν προβλημάτων. Αυτό γίνεται ακόμα πιο σημαντικό, όσο μεγαλώνει ένας ιστότοπος σε όγκο περιεχομένου, οπότε συνήθως χρειάζεται η ύπαρξη τεχνικά καταρτισμένου προσωπικού για την υποστήριξή του.
\item Αντίθετα με το \textlatin{WordPress} δεν παρέχεται η δυνατότητα δωρεάν φιλοξενίας και πρέπει να συνυπολογιστεί ένα επιπρόσθετο κόστος για αυτού του είδους τις υπηρεσίες.
\end{itemize}

\subsection{\textlatin{Joomla} (Κάτι μεταξύ \textlatin{WordPress \& Drupal})}
Το \textlatin{Joomla} είναι το δεύτερο σε δημοφιλία \textlatin{CMS}. Οι δυνατότητες του και η ευκολία χρήσης αποτελούν ένα συνδυασμό των υπόλοιπων δύο \textlatin{CMS}. Παρέχει αρκετές δυνατότητες ώστε να μπορεί να χρησιμοποιηθεί για τη δημιουργία και τη συντήρηση της πλειονότητας των ιστότοπων, χωρίς ιδιαίτερο πρόβλημα και χωρίς την πολυπλοκότητα του \textlatin{Drupal}.
Τα πλεονεκτήματά του περιλαμβάνουν:
\begin{itemize}
\item Ιδανικό για κοινωνική δικτύωση. Το \textlatin{Joomla} μπορεί με μεγάλη ευκολία να εισάγει λειτουργικότητα κοινωνικής δικτύωσης σε έναν ιστότοπο.
\item Εύκολη δημιουργία μικρών ιστοτόπων ηλεκτρονικών αγορών. Το \textlatin{Joomla} κάνει τη διαδικασία δημιουργίας ιστοτόπων ηλεκτρονικών αγορών εντελώς απλή και γρήγορη. Και τα άλλα δυο \textlatin{CMS} μπορούν να πετύχουν ανάλογα αποτελέσματα, αλλά συνήθως με περισσότερη προσπάθεια και παραμετροποίηση.
\item Δεν απαιτεί ιδιάτερες τεχνικές γνώσεις. Το \textlatin{Joomla} έχει βρει τη χρυσή τομή μεταξύ της ευκολίας διαχείρησης του \textlatin{WordPress} και της δυναμικότητας του \textlatin{Drupal}. Είναι δυνατή η δημιουργία και η υποστήριξη ενός πολύπλοκου ιστότοπου, χωρίς ιδιαίτερες τεχνικές γνώσεις.
\item Υποστήριξη από την κοινότητα. Το \textlatin{Joomla} παρέχει τη δυνατότητα παροχής δωρεάν βοήθειας μέσω του εξειδικευμένου \textlatin{help portal}. Δεν είναι τόσο γρήγορη η παροχή απαντήσεων ή τόσο εκτεταμένη η γνωσιακή βάση, όπως η αντίστοιχη του \textlatin{Wordpress}, αλλά ούτε επί πληρωμή όπως οι περισσότερες λύσεις παροχής βοήθειας του \textlatin{Drupal}.
\item Δωρεάν. Όπως ακριβώς το \textlatin{Drupal} και το \textlatin{WordPress}.
\end{itemize}

Στα μειονεκτήματα του \textlatin{Joomla} συγκαταλέγεται το γεγονός ότι, όπως και στο \textlatin{Drupal}, δεν παρέχεται η δυνατότητα δωρεάν φιλοξενίας ιστοτόπων.

\subsection{Τελικά Συμπεράσματα Συγκριτικού των \textlatin{CMS}}
Και οι τρεις επιλογές είναι αρκετά καλές στην πλειονότητα των περιπτώσεων. Παρόλα αυτά υπάρχει πάντα μια πιθανότητα μια επιλογή να ταιριάζει καλύτερα σε ένα σενάριο, από ότι μια άλλη~\cite{make_a_website_hub_2017}. Εάν κάποιος θέλει να ξεκινήσει έναν ιστότοπο σχετικά γρήγορα, τότε το \textlatin{Wordpress} είναι η καλύτερη λύση. Αν υπάρχει η πιθανότητα ο ιστότοπος κάποια στιγμή να μεγεθυνθεί μελλοντικά, με απαίτηση για εξελιγμένες δυνατότητες και χαρακτηριστικά, τότε το \textlatin{Drupal} είναι η καλύτερη επιλογή. Ανάμεσα στις δύο αυτές επιλογές βρίσκεται το \textlatin{Joomla}, το οποίο εμφανίζεται καλύτερη επιλογή για ιστότοπους κοινωνικής δικτύωσης ή μικρά ηλεκτρονικά καταστήματα.

Συνοψίζοντας μπορούμε να καταλήξουμε στα εξής αδρά συμπεράσματα:
\begin{itemize}
\item \textlatin{WordPress} – Η καλύτερη επιλογή για αρχάριους, λόγω της ευκολίας χρήσης του. Δουλεύει εξαιρετικά καλά για μικρούς έως μεσαίους ιστότοπους, \textlatin{blogs} και μικρά \textlatin{e-commerce} καταστήματα.
\item \textlatin{Joomla} – Είναι πολύ καλή επιλογή για ιστότοπους κοινωνικής δικτύωσης, αλλά απαιτεί μια βασική κατανόηση και κατοχή τεχνικών γνώσεων.
\item \textlatin{Drupal} – Είναι το πιο δύσκολο από τα τρία, αλλά επίσης και το πιο ισχυρό εργαλείο. Απαιτεί μια σχετική οικειότητα με γλώσσες όπως οι \textlatin{HTML, CSS and PHP}.
\end{itemize}

Στον παρακάτω πίνακα~\ref{tab01}, συγκρίνονται τα τρία παραπάνω \textlatin{CMS} με βάση τα διάφορα ετερογενή χαρακτηριστικά τους~\cite{comparison_tables}.

\selectlanguage{english}
\tiny
\begin{longtable}[H]{|| p{.3\linewidth} | p{.19\linewidth} | p{.19\linewidth} | p{.19\linewidth} ||}
\hline
\textbf{Atribute} & \textbf{Drupal} & \textbf{Joomla} & \textbf{WordPress} \\
\hline\hline \endhead
Website & drupal.org & joomla.org & wordpress.org \\
Latest version & 8.1.3 & 3.6 & 4.6.1 \\
Release date & 2016 Jun 15 & 2016 Aug 4 & 2016 Sep 7 \\
License & Open Source & Open Source & Open Source \\
Supported databases & MySQL, PostgreSQL & MySQL, PostgreSQL, SQL Server & MySQL \\
Platform & PHP & PHP & PHP \\
Security Captcha & No (Plugin Only) & No (Plugin Only) & No (Plugin Only) \\
Content Approval & Yes & Yes & Yes \\
Email Verification & Yes & Yes & Yes \\
Granular Privileges & Yes & Yes & Yes \\
Authentication methods & LDAP (plugin), NTLM (plugin), Custom & LDAP, Custom & LDAP (plugin), Custom \\
Session Management & Yes & Yes & No (Plugin Only) \\
SSL Compatible & Yes & Yes & Yes \\
Login History & Yes & Yes & No (Plugin Only) \\
Modifications History & Yes & No (Plugin Only) & No (Plugin Only) \\
Commercial Support & Yes & Yes & Yes \\
Developer Community & Yes & Yes & Yes \\
Public Forum & Yes & Yes & Yes \\
Plugin API & Yes & Yes & Yes \\
Drag \& Drop Content & No (Plugin Only) & Yes & Yes \\
Image Resizing & No (Plugin Only) & Yes & Yes \\
Multiple Upload & No (Plugin Only) & Yes & Yes \\
Spellchecker & No (Plugin Only) & No (Plugin Only) & Yes \\
Style Wizard & No (Limited) & No & No \\
Subscriptions & No (Plugin Only) & No (Plugin Only) & No (Plugin Only) \\
Template Language & Yes (Limited) & Yes & No \\
Undo & Yes (Limited) & No & Yes (Limited) \\
WYSIWYG Editor & No (Plugin Only) & Yes & Yes \\
Extensible User Profiles & Yes & Yes & No (Plugin Only) \\
Interface Localization & Yes & Ye & Yes \\
Performance \& Caching & Yes & Yes & No (Plugin Only) \\
Load Balancing & Yes (Limited) & Yes & Yes (Limited) \\
Database Replication & Yes (Limited) & Yes & No (Plugin Only) \\
Static Content Export & No & No & No (Plugin Only) \\
Multilingual Content & Yes & No (Plugin Only) & No (Plugin Only) \\
Multi-Site Deployment & Yes & No (Plugin Only) & Yes \\
RSS (Content Syndication) & Yes & Yes & Yes \\
Advertising Management & No (Plugin Only) & Yes & No \\
Content Scheduling & No (Plugin Only) & Yes & Yes (Limited) \\
Inline Administration & Yes & Yes & No (Plugin Only) \\
Package Deployment & No & No & No \\
Sub-sites / Roots & Yes & Yes & Yes \\
Themes / Templates & Yes & Yes & Yes \\
Web Statistics & Yes & Yes & No (Plugin Only) \\
Web-based Translation Management & Yes & No (Plugin Only) & Yes (Limited) \\
Workflow Engine & Yes (Limited) & No (Plugin Only) & No \\
FTP Support & Yes (Limited) & Yes & No (Plugin Only) \\
UTF-8 Support & Yes & Yes & Yes \\
WebDAV Support & No & No & No \\
XHTML Compliant & Yes & Yes & Yes \\
Blog & Yes & Yes & Yes \\
Chat & No (Plugin Only) & No (Plugin Only) & No (Plugin Only) \\
Classifieds & No (Plugin Only) & No (Plugin Only) & No (Plugin Only) \\
Contact Management & No (Plugin Only) & Yes & No (Plugin Only) \\
Forum (Discussion) & Yes & No (Plugin Only) & No (Plugin Only) \\
Document Management & Yes (Limited) & No (Plugin Only) & No \\
Events Management & No (Plugin Only) & No (Plugin Only) & No (Plugin Only) \\
FAQ Management & Yes & Yes & No (Plugin Only) \\
File Distribution & No (Plugin Only) & No (Plugin Only) & No (Plugin Only) \\
Graphs and Charts & No & No (Plugin Only) & No \\
Guestbook & No (Plugin Only) & No (Plugin Only) & No (Plugin Only) \\
Help Desk / Bug Reporting & No (Plugin Only) & No (Plugin Only) & No \\
Job Postings & No (Plugin Only) & No (Plugin Only) & No (Plugin Only) \\
Link Management & No (Plugin Only) & Yes & Yes \\
Mail Form & No (Plugin Only) & Yes & No (Plugin Only) \\
Matrix & No & No & No \\
My Page / Dashboard & No (Plugin Only) & Yes & Yes \\
Newsletter management & No (Plugin Only) & No (Plugin Only) & No (Plugin Only) \\
Photo Gallery & No (Plugin Only) & No (Plugin Only) & Yes \\
Project Tracking & No (Plugin Only) & No (Plugin Only) & No \\
Search Engine & Yes & Yes & Yes \\
Polls & Yes & Yes & No (Plugin Only) \\
Tests / Quizzes & No (Plugin Only) & No (Plugin Only) & No (Plugin Only) \\
Surveys & No (Plugin Only) & No (Plugin Only) & No (Plugin Only) \\
Time Tracking & No (Plugin Only) & No (Plugin Only) & No (Plugin Only) \\
WYSIWYG & No (Plugin Only) & Yes & Yes \\
User Contributions & Yes & Yes & Yes \\
Web Services Front End & Yes (Limited) & Yes & No (Plugin Only) \\
Wiki & No (Plugin Only) & No (Plugin Only) & No (Plugin Only) \\
Shopping Cart & No (Plugin Only) & No (Plugin Only) & No (Plugin Only) \\
SEO Metadata & Yes & Yes & Yes \\
SEO Friendly URLs & Yes & Yes & Yes \\
Site Map & No (Plugin Only) & No (Plugin Only) & No (Plugin Only) \\
\hline
\caption{\textgreek{Πίνακας Σύγκρισης} Open Source CMS}
\label{tab01}
\end{longtable}
\normalsize
\selectlanguage{greek}

Στη συνέχεια θα παρουσιαστεί βήμα - βήμα, η διαδικασία εγκατάστασης και δημιουργίας ενός απλού ιστότοπου με τη βοήθεια του \textlatin{Drupal}.

\chapter{Ανάπτυξη Ιστοσελίδας \textlatin{Drupal}}\label{ch3}
Για την ανάπτυξη της ιστοσελίδας της εργασίας χρησιμοποιήθηκε το εργαλείο αυτοματοποίησης \textlatin{vagrant}, το οποίο βρίσκεται διαθέσιμο δωρεάν στο Διαδίκτυο (\textlatin{Open Source Software}). Στα επόμενα γίνεται μια σύντομη αναφορά στο υπόψη εργαλείο και στη διαδικασία της ανάπτυξης.

\section{\textlatin{\textlatin{Vagrant (Open Source VM Provissioner)}}}\label{vagrant}
Το \textlatin{Vagrant} είναι ένα εργαλείο δημιουργίας και διαχείρησης εικονικών μηχανών με τη χρήση μιας εξαιρετικά απλοποιημένης διαδικασίας~\cite{vagrant_by_hashicorp}. Το εργαλείο αυτό δίνει έμφαση στην αυτοματοποιημένη διαχείριση των εικονικών μηχανών και μειώνει σημαντικά το χρόνο δημιουργίας και παραμετροποίησης ενός \textlatin{development server}.

Είναι γραμμένο στη γλώσσα προγραμματισμού \textlatin{Ruby} και αποτελεί έναν ενιαίο τρόπο επικοινωνίας με δίάφορους \textlatin{providers} εικονικών μηχανών (όπως \textlatin{VirtualBox, VMware, AWS} κ.α.). Με τον τρόπο αυτό είναι δυνατή η δημιουργία εικονικών μηχανών με τις επιθυμητές παραμέτρους στον μικρότερο δυνατό χρόνο. Παράλληλα, για την εγκατάσταση πακέτων λογισμικού αλλά και παραμετροποίηση σε επίπεδο λειτουργικού συστήματος (ΛΣ), είναι δυνατή η συνεργασία με ευρέως διαδεδομένα \textlatin{provisioning tools}, όπως \textlatin{Chef, Puppet, Ansible} ακόμα και με απλά \textlatin{shell scripts}.

Το μεγαλύτερο ίσως πλεονέκτημα του υπόψη εργαλείου είναι η δυνατότητα παροχής στους προγραμματιστές ενός ενιαίου περιβάλλοντος, το οποίο είναι σταθερό και όσο κοντά γίνεται στο παραγωγικό εξυπηρετητή. Επίσης επειδή η παραμετροποίηση γίνεται με αυτόματο τρόπο, αφαιρείται από τους προγραμματιστές το βάρος της δημιουργίας, συντήρησης και αποσφαλμάτωσης του περιβάλλοντος ανάπτυξης.

Η αρχή λειτουργίας του \textlatin{Vagrant} στηρίζεται στην ύπαρξη μιας εικονικής μηχανής στελέχους (\textlatin{template/vagrant box}), η οποία είναι διαθέσιμη από τα επίσημα αποθετήρια \textlatin{\url{https://vagrantcloud.com/boxes/search}} είτε μπορεί να είναι δική μας. Κατόπιν μέσω μιας διαδικασίας κλωνοποίησης και εφαρμογής παραμέτρων, εντελώς διαφανούς για το χρήστη, αποδίδεται η εικονική μηχανή.

Όλα τα παραπάνω γίνονται με την εκτέλεση της εντολής \textlatin{\texttt{vagrant}} ακολουθούμενης από το αντίστοιχο \textlatin{switch}. Για παράδειγμα, η παρακάτω ακολουθία εντολών κατεβάζει μια εικονική μηχανή \textlatin{ubuntu 64bit} από το επίσημο αποθετήριο και την θέτει σε λειτουργία με τη βοήθεια του \textlatin{VirtualBox}.
\selectlanguage{english}
\scriptsize
\begin{verbatim}
  $ vagrant box add ubuntu/xenial64
  $ vagrant init
  $ vagrant up --provider=virtualbox
\end{verbatim}
\normalsize
\selectlanguage{greek}

Για τη φιλοξενία του ιστοτόπου της εργασίας χρησιμοποιήθηκε μια μηχανή \textlatin{Centos7 64bit} από το επίσημο αποθετήριο. Επειδή η ανάπτυξη έγινε σε \textlatin{Fedora Linux}, χρησιμοποιήθηκε ως \textlatin{Virtualization provider} το παρεχόμενο από το ίδιο το λειτουργικό \textlatin{KVM / libvirt}. Στην συνέχεια με τη χρήση \textlatin{shell provissioner} έγινε η εγκατάσταση και παραμετροποίηση της βάσης δεδομένων (\textlatin{mariadb}) και του \textlatin{webserver (Apache 2.4, php 5.6)}. Με τη χρήση του ίδιου \textlatin{provissioner} συγχρονίστηκε ο κώδικας και τέθηκαν τα σωστά \textlatin{filesystem permissions}.
Όλα τα παραπάνω ορίζονται στο αρχείο \textlatin{\texttt{Vagrantfile}} το οποίο παρατίθεται στο Παράρτημα~\ref{AppA}.

\section{Εγκατάσταση Ιστοτόπου \textlatin{Drupal} Βήμα - Βήμα}\label{manual_setup}
Η διαδικασία ξεκινάει με τη δημιουργία της εικονικής μηχανής, όπως περιγράφηκε στο~\ref{vagrant}. Παίρνουμε κονσόλα στο μηχάνημα με την εντολή \textlatin{\texttt{vagrant ssh}} και έπειτα ενημερώνουμε το σύστημα με τα τελευταία πακέτα.
\selectlanguage{english}
\scriptsize
\begin{verbatim}
  # yum update
\end{verbatim}
\normalsize
\selectlanguage{greek}

Στη συνέχεια εγκαθίσταται το λεγόμενο \textlatin{LAMP stack (Linux, Apache, MySQL, PHP)}. Μαζί εγκαθίσταται και η βιβλιοθήκη γραφικών της \textlatin{php, GD}. Ενεργοποιούνται οι σωστές ζώνες ώρας στο σύστημα και στην \textlatin{php}, καθώς και οι υπηρεσίες του \textlatin{webserver} και της βάσης δεδομένων, ώστε να εκκινούν κατά την έναρξη του συστήματος.
\selectlanguage{english}
\scriptsize
\begin{verbatim}
  # yum install -y epel-release
  # yum install -y http://dl.iuscommunity.org/pub/ius/stable/CentOS/7/x86_64/
  ius-release-1.0-14.ius.centos7.noarch.rpm
  # yum install -y policycoreutils-python wget httpd mariadb-server php56u php56u-pdo php56u-mysqlnd
  php56u-gd php56u-mysql php56u-opcache php56u-mbstring
  # timedatectl set-timezone Europe/Athens
  # sed -i 's/;date.timezone =/date.timezone = Europe\/Athens/g' /etc/php.ini
  # systemctl enable httpd
  # systemctl enable mariadb
\end{verbatim}
\normalsize
\selectlanguage{greek}

Στη συνέχεια γίνεται λήψη του αντίστοιχου πακέτου της \textlatin{Drupal}.
\selectlanguage{english}
\scriptsize
\begin{verbatim}
  # mkdir -p /var/www/html/example.com/public_html/sites/default
  # cd /var/www/html/example.com
  # wget http://ftp.drupal.org/files/projects/drupal-8.0.5.tar.gz
  # tar -zxvf drupal-8.0.5.tar.gz --strip-components=1 -C public_html
  # chown -R apache:apache /var/www/html/example.com
\end{verbatim}
\normalsize
\selectlanguage{greek}

Τα αρχεία ρυθμίσεων του \textlatin{Drupal} θα γραφτούν με τις σωστές ρυθμίσεις όταν ξεκινήσει ο οδηγός παραμετροποίησης. Τα αρχεία αυτά όμως πρέπει να δημιουργηθούν από πριν και να έχουν τα σωστά \textlatin{filesystem permissions}, ώστε να μπορεί ο οδηγός να γράψει σε αυτά.
\selectlanguage{english}
\scriptsize
\begin{verbatim}
  # cd /var/www/html/example.com/public_html/sites/default
  # cp default.settings.php settings.php && sudo cp default.services.yml services.yml
  # chmod 666 {services.yml,settings.php}
\end{verbatim}
\normalsize
\selectlanguage{greek}

Στη συνέχεια θα πρέπει να οριστούν τα \textlatin{trusted hostnames} τα οποία θα χρησιμοποιούν οι χρήστες για να έχουν πρόσβαση στον ιστότοπο (προαιρετικά). Για αυτό το λόγο τροποποιούμε το αρχείο \textlatin{\texttt{/var/www/html/example.com/public\_html/sites/default/settings.php}}, όπως παρακάτω:
\selectlanguage{english}
\scriptsize
\begin{verbatim}
  $settings['trusted_host_patterns'] = array(
    '^example\.com$',
    '^.+\.example\.com$',
  );
\end{verbatim}
\normalsize
\selectlanguage{greek}

Με αυτό τον τρόπο ορίζουμε ότι ο εξυπηρετητής θα σερβίρει προς το Διαδίκτυο οποιοδήποτε ιστότοπο του ορίσουμε με \textlatin{fqdn}, το οποίο τελειώνει σε \textlatin{example.com}.

Εάν θέλουμε να χρησιμοποιήσουμε \textlatin{cleanURL} στον ιστότοπό μας (\textlatin{url} της μορφής \textlatin{\texttt{http://www.example.com/admin}} αντί της μορφής \textlatin{\texttt{http://www.example.com/?q=admin}}), τότε τροποποιούμε το αρχείο \textlatin{\texttt{/etc/httpd/conf/httpd.conf}}.
\selectlanguage{english}
\scriptsize
\begin{verbatim}
  <Directory /var/www/>
  Options Indexes FollowSymLinks
  AllowOverride All
  Require all granted
    RewriteEngine on
      RewriteBase /
      RewriteCond %{REQUEST_FILENAME} !-f
      RewriteCond %{REQUEST_FILENAME} !-d
      RewriteCond %{REQUEST_URI} !=/favicon.ico
      RewriteRule ^ index.php [L]
  </Directory>
\end{verbatim}
\normalsize
\selectlanguage{greek}

Μετά ακολουθεί η παραμετροποίηση της εγκατάστασης μέσα από τον αντίστοιχο οδηγό. Αρχικά πηγαίνουμε μέσω \textlatin{browser} στην \textlatin{ip} ή \textlatin{hostname} που έχουμε ορίσει για την εικονική μηχανή μας και επιλέγουμε την επιθυμητή γλώσσα και συνεχίζουμε στο επόμενο βήμα.
\begin{figure}[H]
\centering
\includegraphics[width=0.9\textwidth, height=7cm]{drupal-choose-language-gray}
\caption{\textlatin{Drupal 8}, Επιλογή Γλώσσας}
\label{fig:drupal_lang}
\end{figure}

Έπειτα επιλέγουμε το επιθυμητό προφίλ εγκατάστασης.
\begin{figure}[H]
\centering
\includegraphics[width=0.9\textwidth, height=7cm]{drupal-choose-installation-profile-gray}
\caption{\textlatin{Drupal 8}, Επιλογή Προφίλ εγκατάστασης}
\label{fig:drupal_profile}
\end{figure}

Μετά σειρά έχει η παραμετροποίηση της βάσης δεδομένων:
\begin{figure}[H]
\centering
\includegraphics[width=0.9\textwidth, height=8cm]{drupal-database-configuration-gray}
\caption{\textlatin{Drupal 8}, Ρύθμιση της Βάσης Δεδομένων}
\label{fig:drupal_db}
\end{figure}

Σε αυτό το σημείο δίνουμε τις παραμέτρους που αφορούν στο όνομα της βάσης δεδομένων, το όνομα χρήστη και το κωδικό πρόσβασης για τη βάση δεδομένων του ιστότοπου. Στην επόμενη φόρμα του οδηγού δημιουργούμε τον \textlatin{admin} λογαριασμό για τον ιστότοπο. Προσοχή, ο συγκεκριμένος χρήστης δε θα πρέπει να συγχέεται με το χρήστη της βάσης δεδομένων, που ορίστηκε στην προηγούμενη φόρμα.
\begin{figure}[H]
\centering
\includegraphics[width=0.9\textwidth, height=10cm]{drupal-site-configuration-gray}
\caption{\textlatin{Drupal 8}, Ρύθμιση του \textlatin{admin} Λογαριασμού}
\label{fig:drupal_admin}
\end{figure}

Τέλος, μεταφερόμαστε στο \textlatin{admin panel}, όπου μας καλωσορίζει το μήνυμα επιτυχούς εγκατάστασης του \textlatin{Drupal 8}.
\begin{figure}[H]
\centering
\includegraphics[width=0.9\textwidth, height=7cm]{drupal-gray}
\caption{\textlatin{Drupal 8}, Επιτυχής Εγκατάσταση}
\label{fig:drupal_success}
\end{figure}

Εφόσον το \textlatin{Drupal} τελειώσει την παραμετροποίηση μπορούμε να επαναφέρουμε τα σωστά \textlatin{filesystem permissions} στα αρχεία ρυθμίσεων για λόγους ασφαλείας.
\selectlanguage{english}
\scriptsize
\begin{verbatim}
  # chmod 644 /var/www/html/example.com/public_html/sites/default/{settings.php,services.yml}
\end{verbatim}
\normalsize
\selectlanguage{greek}

Όλες οι παραπάνω διαδικασίες, πλην του οδηγού εγκατάστασης, αυτοματοποιούνται με τη βοήθεια του \textlatin{Vagrantfile} του Παραρτήματος~\ref{AppA}, όπως περιγράφηκε στο οικείο τμήμα~\ref{vagrant}.

Σειρά από εδώ και εμπρός έχει η επιλογή θέματος (προαιρετικά) και η προσθήκη περιεχομένου στη νέα ιστοσελίδα που δημιουργήθηκε.

\section{Προσθήκη Περιεχομένου στο \textlatin{Drupal} Ιστότοπο}
Από το μενού \textlatin{\texttt{Appearence -> Install new theme}} επιλέγουμε το θέμα, το οποίο θέλουμε να εφαρμόσουμε στο νέο μας ιστότοπο. Για τις ανάγκες της εργασίας επιλέξαμε ένα δωρεάν θέμα \textlatin{bloging}. Η διαδικασία επιλογής θέματος φαίνεται στο Σχήμα~\ref{fig:drupal_theme}.
\begin{figure}[H]
\centering
\includegraphics[width=0.9\textwidth, height=7cm]{drupal-add-theme-gray}
\caption{\textlatin{Drupal 8}, Επιλογή Θέματος}
\label{fig:drupal_theme}
\end{figure}

Μετά την εφαρμογή του θέματος, μπορούμε να εισάγουμε περιεχόμενο στον νέο μας ιστότοπο. Στις επόμενες εικόνες παρουσιάζεται η φόρμα εισαγωγής νέου περιεχομένου, η οποία προκύπτει από το μενού \textlatin{\texttt{Content -> Add new content}} (Σχήμα~\ref{fig:drupal_add_content}).
\begin{figure}[H]
\centering
\includegraphics[width=0.9\textwidth, height=9cm]{drupal-add-content-gray}
\caption{\textlatin{Drupal 8}, Φόρμα Εισαγωγής Νέου Περιεχομένου}
\label{fig:drupal_add_content}
\end{figure}

Το τελικό αποτέλεσμα εμφανίζεται στο Σχήμα~\ref{fig:drupal_blog_entry}
\begin{figure}[H]
\centering
\includegraphics[width=0.9\textwidth, height=9cm]{drupal-blog-entry-gray}
\caption{\textlatin{Drupal 8}, Νέο Άρθρο στο \textlatin{blog}}
\label{fig:drupal_blog_entry}
\end{figure}

Με παρόμοιο τρόπο μπορούμε να προσθέτουμε νέο περιεχόμενο και να ανανεώνουμε συνεχώς τον ιστότοπο.

Αξίζει να σημειωθεί ότι το θέμα που χρησιμοποιήθηκε, παρέχει πλήρη λειτουργικότητα \textlatin{bloging} με δυνατότητα εγγραφής χρηστών, ανάρτησης σχολίων, \textlatin{RSS feed} κ.α. Επίσης όλα τα λογισμικά που χρησιμοποιήθηκαν για τη δημιουργία του ιστότοπου ήταν δωρεάν ανοιχτού κώδικα, κάτι που κάνει τις συγκεκριμένες τεχνολογίες του Διαδικτύου ανοιχτές προς όλους, χωρίς κόστος.

\section{Συμπεράσματα}
Στο παρόν, παρουσιάστηκε η διαδικασία ανάπτυξης καθώς και η λειτουργικότητα μιας απλής ιστοσελίδας με τη χρήση του \textlatin{Drupal CMS}. Έγινε αναφορά σε βασικές έννοιες των συστημάτων διαχείρησης περιεχομένου, με τα πλεονεκτήματα και τα μειονεκτήματά τους. Στη συνέχεια περιγράφηκαν τα πιο γνωστά \textlatin{CMS} ανοιχτού κώδικα, με μια αναφορά στις κύριες δυνατότητες και αδυναμίες τους. Επιλέχθηκαν τα τρία πιο γνωστά \textlatin{CMS} και υπεβλήθησαν σε αναλυτικό συγκριτικό από το οποίο προέκυψαν χρήσιμα συμπεράσματα.

Έπειτα, έγινε αναφορά στα εργαλεία λογισμικού ανοικτού κώδικα, τα οποία χρησιμοποιήθηκαν για την ανάπτυξη του ιστοτόπου της εργασίας και ο τρόπος με τον οποίο αυτοματοποιήθηκαν και συνδυάστηκαν, για την παραγωγή του τελικού αποτελέσματος, μετά και τη διαδικασία αποσφαλμάτωσης. Τέλος, παρουσιάστηκαν οι βασικές λειτουργίες της σελίδας με συνοπτικό τρόπο και τη βοήθεια ανάλογων \textlatin{screenshots}.

Η ιστοσελίδα της εργασίας παρέχει πλήρη λειτουργικότητα \textlatin{bloging} με δυνατότητα εγγραφής χρηστών, ανάρτησης σχολίων, \textlatin{RSS feed} κ.α. Από την ανάλυση που προηγήθηκε προκύπτει η ευκολία με την οποία μπορεί, ακόμα και ένας μη τεχνικός χρήστης, να χρησιμοποιήσει δωρεάν εργαλεία ανοιχτού κώδικα, έτσι ώστε να δημιουργήσει ένα απλό ιστότοπο μέσα σε λίγα λεπτά. Το παράδειγμα της εργασίας χρησιμοποίησε το \textlatin{Drupal} για τη δημιουργία του ιστότοπου. Ένας μη τεχνικός χρήστης μπορεί να φτιάξει έναν ιστότοπο με πιο απλό τρόπο, χρησιμοποιώντας άλλα \textlatin{CMS}, π.χ. το \textlatin{WordPress}, όπως περιγράφηκε στο τμήμα~\ref{ch2}.

\begin{appendices}
\chapter{Αρχείο Ρύθμισης Εικονικής Μηχανής \textlatin{Vagrantfile}}\label{AppA}
\selectlanguage{english}
\inputminted[linenos, fontsize=\scriptsize, breaklines, baselinestretch=1]{ruby}{sources/Vagrantfile}
\selectlanguage{greek}
\end{appendices}

\appendix

\bibliographystyle{babplain}
\bibliography{gnu-cms}

\end{document}