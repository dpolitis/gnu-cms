%!TEX options = -shell-escape

\documentclass[12pt]{report}
\usepackage[a4paper,twoside,top=20mm,bottom=20mm,inner=30mm,outer=25mm]{geometry}
\usepackage[utf8]{inputenc}
\usepackage[greek,english]{babel}
\usepackage[scaled=0.86]{couriers}
\usepackage[toc,page,title,titletoc]{appendix}
\usepackage[pdfpagelabels,unicode]{hyperref}
\usepackage{bookmark}
\usepackage[fixlanguage]{babelbib}
\selectbiblanguage{greek}
\usepackage{titlesec}
\usepackage{etoolbox}
\usepackage{graphicx}
\usepackage{array}
\usepackage{amsmath}
\usepackage{minted}
\usepackage{subcaption}
\captionsetup{compatibility=false}
\graphicspath{ {images/} }
\usepackage[noend]{algpseudocode}
\usepackage{algorithm}
\usepackage{afterpage}
\usepackage{comment}
\usepackage{longtable}

\newcommand\blankpage{%
    \null
    \thispagestyle{empty}%
    \addtocounter{page}{-1}%
    \newpage}

\hypersetup{
  colorlinks=true,
  % linkcolor=green,
  citecolor=red,
  % filecolor=blue,
  urlcolor=blue,
  % pdftitle=,
  % pdfauthor=,
  % pdfsubject=,
  % pdfkeywords=
}

\setcounter{secnumdepth}{3}
\setcounter{tocdepth}{3}

\titleformat{\chapter}
  {\normalfont\LARGE\bfseries}{\thechapter}{1em}{}
\titlespacing*{\chapter}{0pt}{3.5ex plus 1ex minus .2ex}{2.3ex plus .2ex}

\makeatletter
\patchcmd\@resets@pp{%
  \def\Hy@chapapp{\appendixname }%
}{%
  \def\Hy@chapapp{appendix}%
}{}{\errmessage{Cannot patch \string\@resets@pp}}
\patchcmd\@resets@ppsub{%
  \def\Hy@chapapp{\appendixname }%
}{%
  \def\Hy@chapapp{appendix}%
}{}{\errmessage{Cannot patch \string\@resets@pp}}
\makeatother

\addto{\captionsgreek}{\renewcommand{\appendixpagename}{Παραρτήματα}}
\addto{\captionsgreek}{\renewcommand{\appendixtocname}{Παραρτήματα}}
\addto{\captionsgreek}{\renewcommand{\appendixname}{Παράρτημα}}

\begin{document}
\selectlanguage{greek}

\hypersetup{pageanchor=false}

\begin{titlepage}
  \centering
  \includegraphics[width=0.15\textwidth]{pyrforos}\par\vspace{1cm}
  {\scshape\LARGE Εθνικό Μετσόβιο Πολυτεχνείο\\
  Σχολή Ηλεκτρολόγων Μηχανικών και Μηχανικών Η/Υ\par}
  \vspace{1cm}
  {\scshape\Large Εργασία στο Μεταπτυχιακό Μάθημα\\
  Τεχνολογίες Λογισμικού Για Παροχή Υπηρεσιών σε Επικοινωνιακά Δίκτυα\par}
  \vspace{1.5cm}
  {\Large\bfseries Συστήματα Διαχείρισης Περιεχομένου Ανοιχτού Κώδικα\par}
  \vspace{2cm}
  {\large Δημήτριος Πολίτης (ΥΔ)\par}
  \vfill
  Επιβλέπων \par
  Καθ. Ευστάθιος Συκάς

  \vfill

% Bottom of the page
  {\large \today\par}
  \afterpage{\blankpage}
\end{titlepage}

\tableofcontents
\thispagestyle{empty}

\listoftables
\thispagestyle{empty}

\listoffigures
\thispagestyle{empty}

\begin{abstract}
Στο παρόν μελετώνται τα λογισμικά ανοιχτού κώδικα, τα οποία αφορούν σε διαχείριση περιεχομένου, με έμφαση στο \textlatin{web content} και τις δυναμικές ιστοσελίδες. Παρουσιάζονται αρχικά τα διαθέσιμα λογισμικά, τα πλεονεκτήματα και μειονεκτήματά τους και στη συνέχεια περιγράφεται αναλυτικά η διαδικασία δημιουργίας ενός ιστοτόπου \textlatin{Drupal} με τη χρήση αυτοματοποιημένων εργαλείων (\textlatin{ansible, vagrant}).

\vspace{10mm}

\noindent \textbf{Λέξεις κλειδιά:} Συστήματα Διαχείρισης Περιεχομένου, Ανοιχτός Κώδικας, Εξυπηρετητής Ιστοσελίδων Διαδίκτυο.
\end{abstract}

\hypersetup{pageanchor=true}
\clearpage
\pagenumbering{arabic}

\chapter{Εισαγωγή}\label{ch1}
\section{Εισαγωγή}
Στην εποχή του διαδικτύου λένε ότι υπάρχει κάποιος ή κάτι όταν έχει ηλεκτρονική παρουσία σε αυτό. Μπορεί αυτό να ακούγεται εν μέρει υπερβολικό, αλλά η εικόνα που παρουσιάζει μια εταιρία ή ένα φυσικό πρόσωπο στο διαδίκτυο επηρεάζει σε μεγάλο βαθμό την φήμη και την αξιοπιστία του.

Η δημιουργία, η συντήρηση και η ανανέωση του δυναμικού περιεχομένου των ιστοτόπων αποτελεί ένα δύσκολο αντικείμενια την διευκόλυνση του προσωπικού, το οποίο ασχολείται συστηματικά με τις παραπάνω εργασίες, έχουν αναπτυχθεί ειδικά εργαλεία, τα οποία ονομάζονται Συστήματα Διαχείρισης Περιεχομένου.

\section{Συστήματα Διαχείρισης Περιεχομένου}
Τα Συστήματα Διαχείρισης Περιεχομένου (\textlatin{Content Management Systems, CMS}) είναι εργαλεία δημιουργίας και διαχείρησης ιστοτόπων. Οι εφαρμογές αυτές εξαλείφουν την ανάγκη συγγραφής κώδικα προγραμματισμού~\cite{linode}. Από τη στιγμή την οποία έχουν αναπτυχθεί εντός των υποδομών, επιτρέπουν σε προσωπικό με μη εξειδικευμένες γνώσεις να διαχειρίζονται το περιεχόμενο των ιστοσελίδων τους.
Η επίπονη διαδικασία ανάπτυξης κώδικα για την ανανέωση των ιστοσελίδων, αντικαθίσταται από μια φιλική προς το χρήστη, διεπαφή.

\subsection{Πλεονεκτήματα}
Υπάρχουν αρκετές δεκάδες διαθέσιμα λογισμικά αυτού του είδους και πολλά από αυτά είναι δωρεάν - ανοιχτού κώδικα. Κάθε \textlatin{CMS} έχει διαφορετικά χαρακτηριστικά, δυνατότητες ή διεπαφή χρήστη. Οι εφαρμογές αυτές είναι διαθέσιμες σχεδόν από τα τέλη της δεκαετίας του 1990 και συνεχίζουν να χρησιμοποιούνται από όλο και περισσότερους χρήστες.

Πολλές φορές, ακόμα και έμπειροι προγραμματιστές προτιμούν την ευκολία ενός \textlatin{CMS} από το να γράψουν τον κώδικα για έναν ιστότοπο από την αρχή. Η χρήση των \textlatin{CMS} διευκολύνει την ταχύτερη ανάπτυξη του ιστοτόπου, εφόσον είναι αυτή δυνατή ακόμα και εντός λίγων ημερών. Παρέχει επίσης τη δυνατότητα παρακολούθησης των αλλαγών στον κώδικα με τη χρήση ενσωματωμένων εργαλείων \textlatin{version control}~\cite{wikipedia_2017:02}.

Όλο και περισσότερα \textlatin{CMS} είναι πλέον λογισμικά ανοιχτού κώδικα. Η ιδιότητά αυτή επιτρέπει στους χρήστες να αναπτύσσουν πρόσθετα (\textlatin{add-ons}) και να τεκμηριώνουν πληρέστερα τις λειτουργίες τους. Επίσης, είναι δυνατή η παροχή βοήθειας, ανταλλαγή απόψεων, σχολίων κτλ μεταξύ χρηστών.

Πολλά από τα \textlatin{CMS} δημιουργούν την δομή της ιστοσελίδας χρησιμοποιώντας θέματα. Τα θέματα βοηθούν στην καλή εμφάνιση της ιστοσελίδας, γεγονός το οποίο με τη σειρά του κάνει τους χρήστες να επιστρέφουν για νέο περιεχόμενο. Η αλλαγή θεμάτων και δομής της ιστοσελίδας είναι αρκετά εύκολη διαδικασία με τη χρήση \textlatin{CMS}. Επίσης πολλά από τα θέματα βασίζονται σε αρχεία \textlatin{CSS} ή \textlatin{HTML}, πράγμα το οποίο δίνει τη δυνατότητα στους προγραμματιστές να τα παραμετροποιούν σύμφωνα με τις ανάγκες τους και να διατηρούν μια σταθερή εμφάνιση σε όλες τις ιστοσελίδες. Τέλος, πολλά λογισμικά \textlatin{CMS} παρέχουν είτε δωρεάν επιπλέον θέματα, είτε επί πληρωμή, αυξάνοντας κατακόρυφα τις δυνατότητες των επιλογών.

Τα περισσότερα \textlatin{CMS} είναι φτιαγμένα ώστε να παρέχουν την απαραίτητη συμβατότητα με άλλα \textlatin{frameworks} ή \textlatin{standards}, αυξάνοντας την παραγωγικότητα και τις δυνατότητές τους~\cite{wikipedia_2017:02}.

Επίσης, τα \textlatin{CMS} έχουν τη δυνατότητα για αναβάθμιση όλων των λειτουργικών μερών τους με τη χρήση απλών οδηγών, φιλικών προς το χρήστη. Τις περισσότερες φορές παρέχεται και η δυνατότητα \textlatin{rollback}, εφόσον η διαδικασία δεν επιτευχθεί με επιτυχία.

Τα \textlatin{CMS} βοηθούν στην απλοποίηση των διαδικασιών και από την πλευρά των μη τεχνικών χρηστών, καθώς δεν απαιτούν κάποια εξειδικευμένη τεχνική γνώση για την ανάρτηση νέου περιεχομένου σε υπάρχουσες ιστοσελίδες, για παράδειγμα ιστολόγια.

Τέλος, τα \textlatin{CMS} παρέχουν ένα συγκεκριμένο επίπεδο ασφαλείας έναντι σε επιθέσεις εκμετάλευσης αδυναμιών στον κώδικα των ιστοσελίδων, καθώς πρόκειται για λογισμικά, τα οποία περνούν από εκτεταμένους ελέγχους κατά την ανάπτυξή τους, πριν την κυκλοφορία τους. Αυτό ισχύει τόσο για τα λογισμικά ανοιχτού κώδικα - τα οποία υποστηρίζονται συνήθως από μια μεγάλη κοινότητα χρηστών - όσο και τα εμπορικά \textlatin{CMS}. Στα παραπάνω αξίζει να προστεθεί το γεγονός ότι τα περισσότερα \textlatin{CMS} παρέχουν ενσωματωμένους μηχανισμούς παρακολούθησης και εξαγωγής συμπςερασμάτων από τα αρχεία καταγραφής.

\subsection{Μειονεκτήματα}
Τα λογισμικά \textlatin{CMS} έχουν ένα κύριο μειονέκτημα το οποίο κυρίως αφορά στον περιορισμό των δυνατοτήτων παραμετροποίησης μιας εφαρμογής, όταν αυτή αναμένεται να λειτουργήσει υπό διαφορετικά περιβάλλοντα. Για παράδειγμα, όταν μια εφαρμογή αναγκαστεί να λειτουργήσει πίσω από ένα πλήθος \textlatin{firewall} η \textlatin{reverse proxy}. Το παραπάνω πρόβλημα συνήθως αντιμετωπίζεται με \textlatin{plugins} στα πιο δημοφιλή \textlatin{CMS} (για παράδειγμα \textlatin{wordpress}).

Ένα ακόμα μειονέκτημα, το οποίο αφορά συνήθως στους διαχειριστές συστημάτων, είναι ότι τα περισσότερα \textlatin{CMS} απαιτούν ειδικές ρυθμίσεις - πολλές φορές μη-ασφαλείς ή εγκατάσταση ειδικών εργαλείων στους παραγωγικούς \textlatin{server} (για παράδειγμα \textlatin{drush - drushx} για το \textlatin{drupal}).

\section{Κατηγοριοποίηση των \textlatin{CMS}}
Τα \textlatin{CMS} μπορούν να κατηγοριοποιηθούν με βάση το περιεχόμενο, το οποίο πραγματεύονται, σε κατηγορίες όπως παρακάτω:

\paragraph{\textlatin{CMS} γενικής χρήσης}. Τα \textlatin{CMS} αυτού του τύπου χρησιμοποιούνται για την κατασκευή ιστοσελίδων γενικού περιεχομένου. Συνήθως περιλαμβάνουν ενημερωτικό ή προωθητικό περιεχόμενο, χωρίς να περιλαμβάνουν αγοραπωλησίες. Τα τρία πιο διαδεδομένα σύγχρονα \textlatin{CMS} είναι λογισμικά ανοιχτού κώδικα, το καθένα με τα πλεονεκτήματα και τις αδυναμίες του.

\paragraph{\textlatin{CMS} ηλεκτρονικού εμπορίου}. Συνήθως χρησιμοποιούνται για τη διαχείριση ηλεκτρονικών καταστημάτων. Αποτελούν στις περισσότερες των περιπτώσεων ολοκληρωμένα συστήματα διαχείρησης ηλεκτρονικών αγορών με λειτουργικότητες που εκτείνονται από την καταχώρηση της επιθυμίας ενός πελάτη στο καλάθι αγορών, μέχρι την εκδοσή τιμολογίων, αποστολή στον πελάτη, ακόμα και παρακολούθηση αποθεμάτων.

\paragraph{\textlatin{CMS} δημοπρασιών}. Αποτελούν συναφή κατηγορία με την παραπάνω με τη διαφορά ότι αφορούν ηλεκτρονικές δημοπρασίες. Πολλές φορές μπορούν κάποια \textlatin{CMS} να ανήκουν και στις δυο κατηγορίες (π.χ. \textlatin{PHP-ProBid}).

\paragraph{\textlatin{CMS} διαχείρησης πολυμέσων (\textlatin{DAMS})}. Αφορούν σε \textlatin{CMS} τα οποία χειρίζονται \textlatin{vlogs} ή σελίδες με πολυμέσα γενικότερα.

Στο παρόν παρουσιάζονται τα κυριότερα \textlatin{CMS} γενικού περιεχομένου, ανοιχτού κώδικα με έμφαση στα δημοφιλέστερα, με βάση το μερίδιο αγοράς του καθενός.

\chapter{Παρουσίαση των \textlatin{Open source CMS}}\label{ch2}
Παρακάτω παρουσιάζονται τα κυριότερα \textlatin{Open Source CMS} με μια συνοπτική περιγραφή και τα πλεονεκτήματα - μειονεκτήματά του.

\section{\textlatin{Drupal}}
To \textlatin{Drupal} είναι ένα ισχυρό και εξελιγμένο \textlatin{CMS framework} το οποίο στηρίζει την λειτουργία του στην γλώσσα \textlatin{php} και στην ύπαρξη μιας βάσης δεδομένων, όπως η \textlatin{MySQL}. To  \textlatin{Drupal} παρέχει ένα ευέλικτο περιβάλλον, το οποίο χρησιμοποιείται για τη διαχείριση ιστοτόπων διαφόρων τύπων και προφίλ. Το \textlatin{Drupal} μπορεί να χρησιμοποιηθεί για τη δημιουργία πλούσιων σε περιεχόμενο ιστοσελίδων με διαδραστικό περιεχόμενο, όπως φόρουμ, \textlatin{blogs} χρηστών και υπηρεσίες ανταλλαγής προσωπικών μηνυμάτων.

To \textlatin{Drupal} χρησιμοποιείται σε αρκετούς γνωστούς διαδικτυακούς τόπους, όπως για παράδειγμα το \textlatin{\url{http://www.weather.com}}. Καθώς βρίσκεται σε ενεργό κύκλο ανάπτυξης, αναβαθμίζεται αρκετά συχνά (σχεδόν κάθε 2 - 4 μήνες). Η εκτεταμένη κοινότητα χρηστών παρακολουθεί τα σχετικά συνέδρια, τα οποία λαμβάνουν χώρα κάθε 2 χρόνια σε Ευρώπη και Αμερική.

Η δύναμη του \textlatin{Drupal} βρίσκεται στην καλά οργανωμένη δομή του. Το \textlatin{Drupal} ξεκινά με ένα βασικό σύνολο αρχείων, το οποίο μπορεί να εμπλουτισθεί στη συνέχεια με διάφορα πρόσθετα, όπως θέματα ή αρθρώματα, που αυξάνουν τη λειτουργικότητά του. Ο αρχικός πυρήνας των αρχείων μπορεί να έχει ένα βασικό σετ από πρόσθετα ή θέματα, τα οποία μπορούν να τροποποιηθούν κατά βούληση.

Πέρα από τις παραπάνω παραμετροποιήσεις, υπάρχει διαθεσιμότητα σε σύνολα από πρόσθετα, τα οποία προορίζονται για συγκεκριμένη χρήση και συνήθως έρχονται με τη μορφή \textlatin{Drupal} διανομών. Διατίθενται για παράδειγμα διανομές για απλές εταιρικές ιστοσελίδες, για ενημερωτικού τύπου - πολυμεσικές ή ακόμα και διανομές γενικής χρήσης \textlatin{community-based}.

\section{\textlatin{Joomla}}
Το \textlatin{Joomla} αναπτύσσεται ενεργά από το 2005 και χρησιμοποιείται σε πολύ γνωστές ιστοσελίδες όπως τις \textlatin{eBay, General Electric, Ikea} κ.α. Το \textlatin{Joomla} συγκριτικά με το \textlatin{Drupal}, το οποίο έχει περισσότερα πρόσθετα και θέματα, εμφανίζεται και αυτό αρκετά ενισχυμένο σε πλήθος πρόσθετων και επεκτάσεων. Εκτός αυτού, η βάση χρηστών και των δύο \textlatin{CMS} είναι αρκετά εκτεταμένη και έτσι είναι αρκετά εύκολη η εύρεση οδηγιών στο διαδίκτυο για αντιμετώπιση διαφόρων προβλημάτων.

Το \textlatin{Joomla} απευθύνεται κυρίως σε χρήστες μέσου επιπέδου γνώσεων. Διαθέτει εργαλεία για την συνεργατική ανάπτυξη ενός ιστοτόπου, αλλά και για κετρικοποιημένη διαχείριση της διαδικασίας ανάπτυξης. Η διαδικασία εγκατάστασης δεν είναι ιδιαίτερα δύσκολη, αλλά η δημιουργία περιεχομένου δεν είναι τόσο εύκολη σε σχέση με το επόμενο προς εξέταση \textlatin{CMS}, το \textlatin{Wordpress}.

\section{\textlatin{Wordpress}}
Το \textlatin{Wordpress} δημιουργήθηκε αρχικά ως μια πλατφόρμα για τη δημιουργία και συντήρηση \textlatin{blogs}. Είναι μέχρι σήμερα, χωρίς αμφιβολία η ευκολότερη στη χρήση πλατφόρμα και ίσως η πιο δημοφιλής. Η ανάπτυξή της ξεκίνησε το 2003 και είναι ενεργή μέχρι σήμερα. Χρησιμοποιείται ήδη σε 60 εκατομμύρια περίπου ιστότοπους με ρυθμό αύξησης περίπου 100.000 νέους ιστότοπους ανά ημέρα.

Το \textlatin{Wordpress} είναι καλύτερο για την δημιουργία στατικού περιεχομένου. Αυτό δε σημαίνει όμως ότι δε μπορεί να χρησιμοποιηθεί και για δυναμικό ή περίπλοκο περιεχόμενο, καθώς η μεγάλη βάση χρηστών του έχει προσθέσει πλήθος λειτουργιών. Σήμερα η ανάπτυξή του καθοδηγείται ενεργά από την εταιρία \textlatin{Automattic}, με την απήχησή του να γίνεται όλο και μεγαλύτερη.


\selectlanguage{english}
\tiny
\begin{longtable}[H]{|| p{.3\linewidth} | p{.19\linewidth} | p{.19\linewidth} | p{.19\linewidth} ||}
\hline
\textbf{Atribute} & \textbf{Drupal} & \textbf{Joomla} & \textbf{WordPress} \\
\hline\hline \endhead
Website & drupal.org & joomla.org & wordpress.org \\
Latest version & 8.1.3 & 3.6 & 4.6.1 \\
Release date & 2016 Jun 15 & 2016 Aug 4 & 2016 Sep 7 \\
License & Open Source & Open Source & Open Source \\
Supported databases & MySQL, PostgreSQL & MySQL, PostgreSQL, SQL Server & MySQL \\
Platform & PHP & PHP & PHP \\
Security Captcha & No (Plugin Only) & No (Plugin Only) & No (Plugin Only) \\
Content Approval & Yes & Yes & Yes \\
Email Verification & Yes & Yes & Yes \\
Granular Privileges & Yes & Yes & Yes \\
Authentication methods & LDAP (plugin), NTLM (plugin), Custom & LDAP, Custom & LDAP (plugin), Custom \\
Session Management & Yes & Yes & No (Plugin Only) \\
SSL Compatible & Yes & Yes & Yes \\
Login History & Yes & Yes & No (Plugin Only) \\
Modifications History & Yes & No (Plugin Only) & No (Plugin Only) \\
Commercial Support & Yes & Yes & Yes \\
Developer Community & Yes & Yes & Yes \\
Public Forum & Yes & Yes & Yes \\
Plugin API & Yes & Yes & Yes \\
Drag \& Drop Content & No (Plugin Only) & Yes & Yes \\
Image Resizing & No (Plugin Only) & Yes & Yes \\
Multiple Upload & No (Plugin Only) & Yes & Yes \\
Spellchecker & No (Plugin Only) & No (Plugin Only) & Yes \\
Style Wizard & No (Limited) & No & No \\
Subscriptions & No (Plugin Only) & No (Plugin Only) & No (Plugin Only) \\
Template Language & Yes (Limited) & Yes & No \\
Undo & Yes (Limited) & No & Yes (Limited) \\
WYSIWYG Editor & No (Plugin Only) & Yes & Yes \\
Extensible User Profiles & Yes & Yes & No (Plugin Only) \\
Interface Localization & Yes & Ye & Yes \\
Performance \& Caching & Yes & Yes & No (Plugin Only) \\
Load Balancing & Yes (Limited) & Yes & Yes (Limited) \\
Database Replication & Yes (Limited) & Yes & No (Plugin Only) \\
Static Content Export & No & No & No (Plugin Only) \\
Multilingual Content & Yes & No (Plugin Only) & No (Plugin Only) \\
Multi-Site Deployment & Yes & No (Plugin Only) & Yes \\
RSS (Content Syndication) & Yes & Yes & Yes \\
Advertising Management & No (Plugin Only) & Yes & No \\
Content Scheduling & No (Plugin Only) & Yes & Yes (Limited) \\
Inline Administration & Yes & Yes & No (Plugin Only) \\
Package Deployment & No & No & No \\
Sub-sites / Roots & Yes & Yes & Yes \\
Themes / Templates & Yes & Yes & Yes \\
Web Statistics & Yes & Yes & No (Plugin Only) \\
Web-based Translation Management & Yes & No (Plugin Only) & Yes (Limited) \\
Workflow Engine & Yes (Limited) & No (Plugin Only) & No \\
FTP Support & Yes (Limited) & Yes & No (Plugin Only) \\
UTF-8 Support & Yes & Yes & Yes \\
WebDAV Support & No & No & No \\
XHTML Compliant & Yes & Yes & Yes \\
Blog & Yes & Yes & Yes \\
Chat & No (Plugin Only) & No (Plugin Only) & No (Plugin Only) \\
Classifieds & No (Plugin Only) & No (Plugin Only) & No (Plugin Only) \\
Contact Management & No (Plugin Only) & Yes & No (Plugin Only) \\
Forum (Discussion) & Yes & No (Plugin Only) & No (Plugin Only) \\
Document Management & Yes (Limited) & No (Plugin Only) & No \\
Events Management & No (Plugin Only) & No (Plugin Only) & No (Plugin Only) \\
FAQ Management & Yes & Yes & No (Plugin Only) \\
File Distribution & No (Plugin Only) & No (Plugin Only) & No (Plugin Only) \\
Graphs and Charts & No & No (Plugin Only) & No \\
Guestbook & No (Plugin Only) & No (Plugin Only) & No (Plugin Only) \\
Help Desk / Bug Reporting & No (Plugin Only) & No (Plugin Only) & No \\
Job Postings & No (Plugin Only) & No (Plugin Only) & No (Plugin Only) \\
Link Management & No (Plugin Only) & Yes & Yes \\
Mail Form & No (Plugin Only) & Yes & No (Plugin Only) \\
Matrix & No & No & No \\
My Page / Dashboard & No (Plugin Only) & Yes & Yes \\
Newsletter management & No (Plugin Only) & No (Plugin Only) & No (Plugin Only) \\
Photo Gallery & No (Plugin Only) & No (Plugin Only) & Yes \\
Project Tracking & No (Plugin Only) & No (Plugin Only) & No \\
Search Engine & Yes & Yes & Yes \\
Polls & Yes & Yes & No (Plugin Only) \\
Tests / Quizzes & No (Plugin Only) & No (Plugin Only) & No (Plugin Only) \\
Surveys & No (Plugin Only) & No (Plugin Only) & No (Plugin Only) \\
Time Tracking & No (Plugin Only) & No (Plugin Only) & No (Plugin Only) \\
WYSIWYG & No (Plugin Only) & Yes & Yes \\
User Contributions & Yes & Yes & Yes \\
Web Services Front End & Yes (Limited) & Yes & No (Plugin Only) \\
Wiki & No (Plugin Only) & No (Plugin Only) & No (Plugin Only) \\
Shopping Cart & No (Plugin Only) & No (Plugin Only) & No (Plugin Only) \\
SEO Metadata & Yes & Yes & Yes \\
SEO Friendly URLs & Yes & Yes & Yes \\
Site Map & No (Plugin Only) & No (Plugin Only) & No (Plugin Only) \\
\hline
\caption{\textgreek{Πίνακας Σύγκρισης} Open Source CMS}
\label{tab01}
\end{longtable}
\normalsize
\selectlanguage{greek}

\section{Συμπεράσματα}
Στο παρόν, παρουσιάστηκε ένα μοντέλο προσομοίωσης του πρωτοκόλλου πολλαπλής πρόσβασης μέσου (\textlatin{MAC}) τύπου \textlatin{slotted aloha}. Τα αποτελέσματα της προσομοίωσης έδειξαν τιμές πολύ κοντά στις θεωρητικά αναμενόμενες. Συγκεκριμένα, η αύξηση των μνημών προσωρινής αποθήκευσης πακέτων (\textlatin{receiving buffers}) στους σταθμούς, είχε θετική επίδραση στην απόδοση του συστήματος \textlatin{S}, η οποία όμως εμφανίζονταν φθίνουσα, καθώς αυξάνονταν το πλήθος των σταθμών \textlatin{M}. Επίσης, η αύξηση του πλήθους των \textlatin{receiving buffers} πάνω από δύο, δεν είχε κάποια σημαντική επίδραση στην αποδοτικότητα του συστήματος. Αντιθέτως, η αύξηση των διαύλων επικοινωνίας μεταξύ των κόμβων είχε σημαντικά θετική επίδραση στην διεκπεραιωτική δυνατότητα του συστήματος και τη μείωση της καθυστέρησης, ανεξάρτητα από το πλήθος των σταθμών.

Τα παραπάνω καταδεικνύουν ότι η βέλτιστη απόδοση του πρωτοκόλλου επιτυγχάνεται όταν οι σταθμοί έχουν δύο το πλήθος \textlatin{receiving buffers} και ικανό αριθμό καναλιών - διαύλων επικοινωνίας μεταξύ τους (ο οποίος μπορεί να περιορίζεται από τεχνικούς περιορισμούς ή περιορισμούς κόστους). Στην περίπτωση των ασύρματων δικτύων αυτό μπορεί να μεταφραστεί: είτε σε ικανό αριθμό πομποδεκτών ανά σταθμό, οι οποίοι θα λειτουργούν σε διαφορετικές συχνότητες για λόγους αποφυγής παρεμβολών, είτε σε κατάλληλο σχήμα πολύπλεξης (\textlatin{TDM, FDM}), επαναχρησιμοποίηση φάσματος με αναπήδηση συχνότητας κ.α. Κατά αντιστοιχία, στα οπτικά δίκτυα είναι δυνατή η χρησιμοποίηση πολύτροπων οπτικών ινών και εκπομπή των δεδομένων σε διαφορετικά μήκη κύματος για την ταυτόχρονη χρησιμοποίηση του μέσου.

\begin{appendices}

\end{appendices}

\appendix

\bibliographystyle{babplain}
\bibliography{gnu-cms}

\end{document}